Quality gates are a set of defined criteria per each step of a software development lifecycle that cannot be
ignored during a transition from one stage to another. They act as control points before (External Quality Gate) and 
after (Internal Quality Gate) each step to monitor and evaluate the project's health by minimizing any sort of errors.
Let's make a concise overview of lifecycle steps and their quality gates (check \cite{Hawl02} for details):

\begin{itemize}
  \item Analysis (External: \emph{Time Management, Risk Management, Project Planning with Quality Assertions}; 
  Internal: -)

  \item Specification (External: \emph{Requirements Management, Prototyping (Proof of Concepts), User Interface, 
  Quality Requirements, Technical Concept}; Internal: \emph{Correctness, Uniqueness, Completeness, Consistency, 
  Restrictions})

  \item Draft (External: \emph{Plausibility, Flexibility, Maintenance Cost, Serviceability}; Internal: 
  \emph{Architecture, Technical Requirements, Interfaces Declaration (Documentation), Test Plan, Integration Plan}) 

  \item Coding (External: \emph{Complexity, Scope, Documentation}; Internal: \emph{Legibility, Structuredness, 
  Robustness, User Interfaces})

  \item Integration (External: \emph{Mean Time to Failure, Availability Error Rate}, Internal: \emph{Integration 
  Strategy and Steps, Quality Characteristics, Recovery Flow})

  \item Implementation (External: \emph{Telemetry, Installation Instructions, Manual Acceptance Tests}, Internal: -)
\end{itemize}

\noindent For the code itself it can be declared: 

\begin{itemize}
  \item Code Quality Gate (following coding standards [by static code analysis tools]) - maintenance cost.
  \item Test Gate (coverage and effectiveness of tests) -- cost of failure.
  \item Security Gate (evaluates for security vulnerabilities and weaknesses) - reputation cost.
  \item Performance Gate (meets the specified performance criteria and does not degrade under certain loads) - churn rate.
  \item Deployment Gate (stability of the deployment process) - upgrade cost.
\end{itemize}

\noindent As a summary, quality gates promote a culture of continuous improvement by providing a feedback on the 
software's quality. By incorporating and automating quality gates, the company is gaining a confidence regarding the
product quality, since it is designed into the product (not discovered as missing or present later).

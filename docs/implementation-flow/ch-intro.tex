% Copyright 2023 The terCAD team. All rights reserved.
% Use of this content is governed by a CC BY-NC-ND 4.0 license that can be found in the LICENSE file.

\markboth{Introduction}{Introduction}

Dear readers, I invite you to dive together with me into a platform-agnostic application development from a zero-state 
position. Meaning, that we (me and, possibly, you) might know nothing about cross-platform development by using
\q{Flutter}-framework with \q{Dart}-language under the hood, do not have an idea for the application, and chosen only a
target market, which to shake (in our case, -- financial accounting).

That book might be misaligned with concepts that you're following, respect, or adepting; by representing my, 
Viachaslau Lyskouski, personal vision, that's why I invite you to argue with me on the page
\href{https://github.com/lyskouski/app-finance/issues}{https://github.com/lyskouski/app-finance/issues}. And who knows,
may be your arguments will change ground ideas of that book, and we'll republish it together. So, here are those 
statements:

\begin{itemize}
    \item Creating a product (rarely it can be finished and never changed then) is similar a pottery workshop flow, 
    by taking amorphous idea and shaping it. Through the time it can be changed an architectural style (homogeneous, 
    layered, multitier, microservices, ...) and structure (monolithic, distributed, hierarchical, ...), even platform 
    (from platform specific to serverless), and language. It's all about being Agile, be adaptive to all external and 
    internal challenges.

    \item Programming is not something about creativity, in its philosophic meaning (\emph{no one, except the author, 
    will be able to accurately obtain the same result if the same initial situation is created}), but mastery 
    (\emph{how accurate, balanced, and far-sighted will be your implementation}).

    \item Instruments (languages, frameworks, libraries, etc.) should be adapted to solve the problem instead of 
    adapting initial ideas to a capability of available / known tools.

    \item Instead of being \q{I}, \q{T}, \q{U}, or \q{M}-shaped, nowadays dictate us to be \q{\_}-shaped 
    (underscore-shaped); technologies are infinitely replacing each other through the time via tick-tock model 
    (evolution / revolution) and the only right way is to constantly extend background fundamentals by being familiar 
    with used instruments.
\end{itemize}

\noindent In case of having any discrepancies with this, the book may cause a rejection; and it's the best time to 
refund. I've started that book without any knowledge in mobile-development, Flutter / Dart, and understanding the needs
of financial accounting. \q{ChartGPT} was my mentor, \q{Midjourney} has helped to generate a few images (including 
covers of that book) and "shared" some design insights.

Going back to the actuals of the book and chosen technologies... In today's fast-paced digital world, businesses strive 
to adopt innovative technologies to streamline their operations and deliver exceptional user experiences. Programs have 
become indispensable tools for organizations seeking to engage with their customers, increase efficiency, and stay 
ahead of the competition. Flutter has emerged as a popular choice for building high-performance, platform-agnostic 
(desktop, mobile, web) applications.

This book, "From Nothing to Market with Flutter: Desktop, Mobile, and Web Distribution" takes us to a comprehensive 
journey through the creation of a robust competitive application by using Flutter with \q{no-server}-architecture (while 
\q{serverless} is mostly associated with cloud-based infrastructure). And go through building a powerful and efficient 
application together.

The primary focus of this book is to demonstrate how to leverage Flutter's rich ecosystem and no-server approach, 
to create a feature-rich (financial accounting) application. Throughout the chapters, you'll explore the various 
stages of an application creation, from conceptualization to deployment.
\\
\noindent Key features of this book:
\begin{itemize}
    \item Understand Flutter architecture, and how it enables cross-platform development. Make an overview of 
    the Dart programming language, which forms the foundation of Flutter applications.

    \item Gain insights into the design principles of an application development and distribution. 
    
    \item Go through User Interface and Experience practices to create an intuitive user interface.

    \item Discover how to incorporate essential features into the application with a forecast release planning.

    \item Explore strategies for securely interacting with external systems and ensuring data integrity.

    \item Master techniques for testing (unit, widget, integration, and performance tests) and debugging practices. 

    \item Configure the application distribution across all platforms, and explore strategies for ongoing 
    maintenance and updates.
\end{itemize}

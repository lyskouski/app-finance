% Copyright 2023 The terCAD team. All rights reserved.
% Use of this content is governed by a CC BY-NC-ND 4.0 license that can be found in the LICENSE file.

Dear readers, I invite you to join me on a journey into a platform-agnostic application development, starting from 
ground zero. Meaning, that we (me and, possibly, you) might know nothing about cross-platform or mobile development, 
\q{Flutter}-framework, \q{Dart}-language, by not having even an idea for the application; ideally, it might 
be chosen a market, which to shake (in our case, -- financial accounting).

This book offers a different perspective on the concepts that you are interested in, inspired by, or practicing; 
based on the author's vision. I welcome you to share your thoughts with me (\emph{and also help me spot any errors 
or typos}) on the repository that contains the sources of this book and an application that we will develop together 
as we read along (\href{https://github.com/lyskouski/app-finance}{https://github.com/lyskouski/app-finance}). And... 
who knows, maybe your insights will enrich the core ideas of this book, and we will publish a new edition together. 
So, here are the statements:

\begin{itemize}
\setlength{\itemsep}{3pt}
\setlength{\parskip}{0pt}
\setlength{\parsep}{0pt}
    \item Creating any program is similar to a pottery workshop flow, by taking amorphous idea and shaping it. Through 
    the time it can be changed an architectural style (homogeneous, layered, multitier, microservices, etc.), structure 
    (monolithic, distributed, hierarchical, etc.), even platform (from platform specific to serverless), and language. 
    It's all about being Agile, being adaptive to all external and internal challenges. Meaning, that the application's 
    revenue justifies any refactoring, including the possibility of re-creating it from scratch.

    \item Programming is not something about creativity, in its pragmatic meaning (\emph{no one, except the author, 
    will be able to accurately obtain the same result if the same initial situation is created}), but mastery 
    (\emph{how accurate, quick, balanced, and far-sighted will be your implementation}).

    \item Instruments (languages, frameworks, libraries, etc.) should be adapted to solve the problem instead of 
    adapting initial ideas to a capability of available / known tools. Start with a problem, use tools to solve it.

    \item Instead of being \q{I}, \q{T}, \q{U}, or \q{M}-shaped, nowadays dictate us to be \q{\_}-shaped 
    (underscore-shaped). While the most scientists maintain a broad outlook with a narrow specialty (\q{T}-shaped), a 
    significant portion of programmers stand on an unstable foundation. Meaning that technologies are infinitely 
    replacing each other through the time via tick-tock model (characterized by periods of evolution and revolution) 
    and the most effective approach is to continuously expand one's fundamental knowledge (polymathy, \cite{Root09}) 
    by staying acquainted with the technologies in use. Since, the Program is no more than Algorithms and Data 
    Structures \cite{Wirt76}.
\end{itemize}

\noindent Besides, this book is written from scratch, without any prior knowledge in mobile and cross-platform 
development, \q{Flutter}-framework, \q{Dart}-language, or the demands of the "financial accounting" market. 
\q{ChatGPT} was used as a mentor, and \q{Midjourney} -- to generate some images (including the cover of this 
book), and to provide some design tips.

Returning to the core themes of the book and the selected technologies, it's evident that businesses aim to embrace 
innovative technologies to optimize their processes and provide outstanding user experiences, all while ensuring 
broad platform compatibility. Not so far, Flutter (\emph{Dec 4th, 2018 -- version 1; May 10th, 2023 -- version 3}) 
has emerged as a popular choice for building high-performance, platform-agnostic (desktop, mobile, web) applications. 
And this book, "\emph{From Zero to Market with Flutter: Desktop, Mobile, and Web Distribution}", road us to a 
comprehensive journey through the creation of a robust competitive application by using Flutter with 
\q{no-server}-basis (while \q{serverless} is mostly associated with cloud-based infrastructure).

The central objective of this book is to illustrate how to harness the extensive capabilities of Flutter to build a 
comprehensive and feature-rich application. Across the chapters, we will delve into every phase of the application 
development process, starting from conceptualization and culminating in distribution:

\begin{itemize}
\setlength{\itemsep}{3pt}
\setlength{\parskip}{0pt}
\setlength{\parsep}{0pt}
  \item Take basics in \q{Dart}-language and \q{Flutter}-framework, and how it enables cross-platform development.

  \item Gain insights into the design principles of an application development. 
  
  \item Go through User Interface and Experience practices to create an intuitive user interface.

  \item Discover how to incorporate essential features into the application with a forecast release planning.

  \item Explore strategies for secure interaction with external systems and ensuring data integrity.

  \item Master techniques for testing (unit, widget, integration, and performance tests) and debugging practices. 

  \item Configure the application distribution across all platforms (Windows, Linux, MacOS, Android, iOS) to their 
  marketplaces, and explore strategies for ongoing maintenance and updates.
\end{itemize}

\noindent So, let's embark on this learning curve journey and keep \q{Flutter}ing.
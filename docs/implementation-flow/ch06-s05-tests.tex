% Copyright 2023 The terCAD team. All rights reserved.
% Use of this content is governed by a CC BY-NC-ND 4.0 license that can be found in the LICENSE file.

\subsection{Conducting Usability Tests} \label{usability}
\markboth{Optimizing}{Conducting Usability Tests}

A usability test is an essential method in user research, offering valuable insights that cannot be obtained from 
quantitative data alone. This form of research delves into users' needs and expectations and the reasons behind their 
behavior. Unlike quantitative data, such as conversion rates, usability tests reveal the reasons behind user actions. 
Various approaches and methodologies can be employed to conduct effective usability tests and gain a deeper 
understanding of user experiences and interactions:

\begin{itemize}
  \item (Remote) Moderated Usability Testing -- a facilitator guides users through predefined tasks. And, unmoderated 
  -- without assistance.

  \item Thinking Aloud -- users vocalize their thoughts and feelings while navigating an app.

  \item A/B Testing -- involves comparing two or more versions of a design to see which performs better. 
  Comparative Usability Testing -- multiple designs or competitors' products are tested to determine which one 
  performs better in terms of usability.

  \item Eye-Tracking Studies -- by a specialized equipment to monitor where users look on a screen to identify visual 
  attention patterns.

  \item Card Sorting and Click Testing -- providing insights into how users expect information to be structured, and 
  which parts are most engaging or confusing.

  \item Accessibility Testing -- involving individuals with disabilities to understand the problem areas of the 
  application; additionally, automate WCAG Accessibility Standards verification.

  \item Surveys and Questionnaires -- collecting user feedback through structured surveys on user satisfaction and 
  preferences.
\end{itemize}

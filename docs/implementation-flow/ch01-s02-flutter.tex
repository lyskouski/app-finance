% Copyright 2023 The terCAD team. All rights reserved.
% Use of this content is governed by a CC BY-NC-ND 4.0 license that can be found in the LICENSE file.

\subsection{Flutter}
\markboth{Bootcamping}{Flutter}

Flutter is an open-source framework developed by Google for building natively compiled applications for mobile, web, and 
desktop from a single codebase by using \q{Dart}-language. The framework provides a comprehensive set of libraries and 
tools for building UIs, handling user input, and accessing device features. Flutter's UI is constructed by using widgets, 
that are the building blocks of Flutter apps, and everything, from a simple button to a complex layout, is a widget
(including animations and layout controls).

\paragraph{Engine} At the core of Flutter is the C/C++ engine, built around \q{Skia}, an open-source 2D graphics 
library maintained by Google. Skia provides low-level APIs for rendering vector and bitmap graphics, including shapes, 
text, and images, along with features such as anti-aliasing for smoother visuals. The engine also bridges to 
platform-specific code to access device hardware and system services. In recent versions, Flutter introduced 
\q{Impeller}, a modern rendering runtime designed as a replacement for Skia in certain contexts. Impeller is now the 
default renderer on iOS (and gradually being rolled out to other platforms) and brings improved performance, 
consistency across devices, and support for advanced rendering features such as 3D and more efficient GPU utilization.

\paragraph{Embedder} The Flutter Engine is window toolkit agnostic, meaning a capability to build Flutter embedders on 
one of the platforms not supported out of the box. That enables creation of custom solutions with the power and 
flexibility of Flutter for a distribution to anywhere, including smart devices, cars, and more (as \q{raspi-flutter} 
for \q{Raspberry Pi}).

\paragraph{Platform Channels} Device-specific features and APIs are accessible through what are known as platform channels, 
which facilitate the integration of a native functionality. Platform Channels enables bi-directional asynchronous 
communication between Dart and the native code of the host platform. It's possible to invoke methods on that channel 
from either the Dart or native side (for example, a native method to access the device's camera). Bi-directional 
communication is covered by a data serialization and deserialization process that ensures that it is in a format that 
both can understand.

\paragraph{Framework} The Flutter Framework includes Material Design and Cupertino, which provide a wide range of user 
interface components.


\subsubsection{Considering Diversity}

\q{Flutter} offers a vast array of out-of-the-box (OOTB) widgets that greatly facilitate the creation of diverse and 
complex user interfaces. These widgets are designed to meet a wide range of application needs, from creating simple 
layouts to developing intricate, feature-rich interfaces. This comprehensive selection of widgets significantly 
accelerates the development process, enabling us to prioritize meeting functional needs over building UI components 
from scratch.

\begin{itemize}
  \item \q{SizedBox} for a size allocation, \q{Container} extends it by styling as padding, margin, and background color.

  \item \q{Row} and \q{Column} arrange child widgets horizontally and vertically accordingly, and \q{DataTable}  
  mixes both.

  \item \q{ListView} for creating scrollable lists of widgets

  \item \q{Stack} allows to overlay widgets on top of one another.

  \item \q{AppBar} enables a navigation on top of the page, \q{BottomNavigationBar} is specific for the bottom 
  layer, for the sidebar it can be used \q{NavigationRail}, \q{Drawer} -- to implement navigation drawers, and 
  \q{SnackBar} for the popping up content.

  \item \q{FlatButton}, \q{RaisedButton}, \q{FloatingActionButton} -- different buttons with customizable visual feedback.

  \item \q{TextField}-widget offers input fields where users can enter text, numbers, or other data. Meanwhile,
  \q{SearchAnchor} manages selections from a list of options.
  
  \item \q{Card} -- for creating material design cards with a consistent and customizable elevation effect.

  \item \q{PageRouteBuilder} enables custom page transitions and animations when navigating between screens.

  \item \q{FutureBuilder} simplifies working with asynchronous operations by rebuilding itself when the future state is 
  resolved.

  \item \q{InkWell} provides a Material-style ripple effect for making widgets tappable and interactive.
\end{itemize}

\noindent And, a lot of others...


\subsubsection{Handling User Input and Gestures}

User input and event handling make Flutter apps interactive and responsive. Whether through taps, clicks, gestures, or 
keyboard shortcuts, they let apps adapt to different devices—touch on mobile, mouse on desktop, or stylus on 
tablets—enhancing usability across platforms.

Flutter offers the \q{GestureDetector} widget to capture gestures such as taps, drags, and scaling. It also supports 
mouse clicks on desktop and web. Text input is commonly handled using the \q{TextField} and \q{TextFormField} widgets, 
while \q{FocusNode} and \q{FocusScope} manage focus and navigation. Lower-level keyboard input can be handled using the 
\q{RawKeyboardListener} widget. Flutter also supports input from game controllers, accessibility tools, and other 
platform-specific devices. The \q{GestureDetector} does not have a visual representation and acts as an invisible 
wrapper around its child widget, capturing user interactions on that child.

\begin{lstlisting}
GestureDetector(
  onTap: () {
    // Handle tap event
  },
  onDoubleTap: () {
    // Handle double-tap event
  },
  onLongPress: () {
    // Handle long-press event
  },
  onPanUpdate: (details) {
    // Handle drag event
  },
  child: Container(
    width: 200,
    height: 200,
    color: Colors.blue,
    child: Center(child: Text('Tap me')),
  ),
);
\end{lstlisting}

\noindent In this example, the \q{GestureDetector} wraps a \q{Container} widget. It listens for various gestures like
tap, double-tap, long-press, and drag (pan update). When a user interacts with the container, the corresponding callback 
functions are triggered, allowing to handle the events accordingly.

\newpage
\subsubsection{Distinguishing States} \label{flutter-state}

A \q{StatelessWidget} in Flutter represents interfaces with an immutable state (once created), requiring a new instance 
for updates; providing suitability for static content like text, icons, or images; and performance advantages due to 
their lack of mutable state, making them generally more memory and processing-efficient.

Where as \q{StatefulWidget} is capable of changing its properties and appearance over a time. It consists of two classes: 
immutable \q{Widget}-class and a corresponding mutable \q{State}-class. Key attributes encompass mutable state stored 
within the associated \q{State}-object and can be changed by the \q{setState}-method. Suits for crafting interactive 
and dynamic user interfaces such as forms and animations. It's worth noting that \q{StatefulWidget} may demand more 
memory due to a potentially frequent rebuilds, but performance itself can be improved by \q{RepaintBoundary}-widget 
usage that tears off (isolate) the render of bounded widgets from predecessor objects rendering.

\begin{lstlisting}
class LoadingWidget extends StatefulWidget {
  const LoadingWidget({super.key});
  @override
  LoadingWidgetState createState() => LoadingWidgetState();
}
class LoadingWidgetState extends State<LoadingWidget> 
    with TickerProviderStateMixin {
  late AnimationController _controller = AnimationController(
      vsync: this,
      duration: const Duration(seconds: 2),
    )..repeat();

  @override
  void dispose() {
    _controller.dispose();
    super.dispose();
  }
  @override
  Widget build(BuildContext context) => AnimatedBuilder(
      animation: _controller,
      builder: (context, child) {
        return CircularProgressIndicator(
          value: _controller.value,
          color: context.colorScheme.inversePrimary,
          strokeWidth: 4,
        );
      },
    );
}
\end{lstlisting}


\subsubsection{Sharing Data}

Inherited Widgets are a way to propagate data down the widget tree from a parent widget to its descendants. The key 
feature is that they automatically rebuild their descendants when the data they provide changes:

\begin{lstlisting}
class AppWrapper extends InheritedWidget {
  final String data;

  AppWrapper({super.key, this.data, super.child});

  static AppWrapper of(BuildContext context) =>
    context.dependOnInheritedWidgetOfExactType<AppWrapper>();

  @override
  bool updateShouldNotify(covariant InheritedWidget oldWidget) {
    // Set if the data has changed and descendants should rebuild
    return data != (oldWidget as MyInheritedWidget).data;
  }
}
\end{lstlisting}

\noindent To use an Inherited Widget, we should wrap our root widget (e.g., MaterialApp or CupertinoApp) with the custom 
Inherited Widget:

\begin{lstlisting}
runApp(
  AppWrapper(data: 'Shared Data', child: MyApp())
);
\end{lstlisting}

\noindent Descendant widgets can access the shared data using the context:

\begin{lstlisting}
final data = AppWrapper.of(context).data;
\end{lstlisting}

\noindent Inherited Widgets can be used to pass data down the widget tree without manually passing it as constructor 
parameters through multiple levels of widgets. This can greatly simplify the code and reduce the need for callback 
functions to communicate between widgets. Also, they minimize unnecessary rebuilds and updates by only rebuilding the 
widget subtree that depends on the data that changed.


\newpage
\subsubsection{Exploring Widget Lifecycles}

Similar to the app lifecycle, Flutter widgets also possess lifecycle methods:

\begin{itemize}
  \item \q{initState} This method is invoked when the widget is inserted into the widget tree. It runs only once per 
  State object and is where you can initialize variables specific to the widget, such as an AnimationController.

  \item \q{didChangeDependencies} This method is triggered immediately after initState and when a State object's 
  dependencies change through InheritedWidget.

  \item \q{didUpdateWidget} When the widget configuration changes, this method is called. An example is when a 
  parent widget passes data to a child widget through the constructor.

  \item \q{build} The build method is called whenever the widget needs to be rebuilt. This can happen after the 
  initState method, didChangeDependencies, didUpdateWidget, or when the state changes through setState.

  \item \q{deactivate} This method is called when the widget is removed from the widget tree, but not permanently. 
  It can be activated again later.

  \item \q{dispose} The dispose method is called when the widget is removed from the widget tree permanently. 
  In this method, you should release any resources held by the widget, such as stopping animations.
\end{itemize}

\noindent These lifecycle methods allow to manage the behavior and state of widgets throughout their existence in the 
widget tree. The list can be extended by specialized widgets' methods, such as those used for handling animations or 
state management: 

\begin{itemize}
  \item \q{AnimationController}: forward (start the animation), reverse.
  \item \q{StreamBuilder}: stream (to retrieve the stream), snapshot (the most recent snapshot of the data).
  \item \q{FutureBuilder}: future (to access the future), snapshot.
  \item \q{ChangeNotifierProvider}: updateShouldNotify (method for deciding if the associated ChangeNotifier has 
  changed), updateShouldNotifyWith (control over when to notify descendants of changes).
\end{itemize}


\newpage
\subsubsection{Explaining Build Context}

\q{BuildContext} is an identifier for a particular widget in the hierarchy and provides access to information about its 
position in the UI structure. Through it, widgets can access data from inherited widgets like \q{Theme}, \q{MediaQuery}, 
\q{Navigator}, etc., without having direct references. Widgets use \q{BuildContext} to determine when to rebuild. 
Inherited widgets like themes or languages are accessed through \q{BuildContext}, enabling widgets to inherit values 
from their ancestors.

Understanding \q{RenderObjects} is pivotal in Flutter, as they work in tandem with widgets to manage the drawing logic.
\q{RenderObjects} undertake tasks like computing sizes, handling translations, managing colors, and essentially 
performing the bulk of the rendering process. This implies that widgets extending a RenderObjectWidget can directly 
paint elements, not merely compose them.

In this context, while widgets remain immutable, the \q{build}-method is frequently called, generating numerous widgets 
each second. Upon the initial creation and insertion of your widget into the widget tree, it initiates the creation of 
an \q{Element} through its \q{createElement}-method. During subsequent rebuilds, the framework assesses whether the 
widget returned by the \q{build}-method can update the existing element at that specific position within the tree. 
However, it's the \q{RenderObjects} that handle the actual rendering of elements on the screen.

So, it's vital to optimize \q{build}-method for efficiency since it might be invoked up to 120 times a second, depending 
on the device's refresh rate, based on the state change of our application. We should avoid using network calls, 
database operations, or other data handling operations in the \q{build}-method. Failing to deliver results within this 
timeframe can cause dropped frames and, consequently, UI interruptions, leading to a poor user experience.

In summary, developers operate with a hierarchy of widgets that describe the UI. Flutter takes the widget tree and 
generates an element tree. When there are changes in the widget tree (due to state changes, user interactions, etc.), 
Flutter performs a reconciliation process. It compares the new widget tree with the previous one to identify differences 
and determine what elements need updating. Based on the element tree, Flutter generates corresponding Render Objects.
They handle the positioning, size, and appearance of UI elements. After layout, the render objects execute the painting 
process, which involves filling pixels on the screen with the appropriate colors and graphics according to the visual 
representation of the UI. The Render Object Tree gets translated into commands for the graphics pipeline, ultimately 
rendering the UI on the device's screen.


\newpage
\subsubsection{Differentiating Keys}

Keys, in the context of Flutter, are indispensable. They act as distinct identifiers for widgets and wield significant 
influence within the Flutter widget hierarchy. With a profound understanding of these keys, we may harness their power 
to improve the effectiveness of created applications.

\begin{itemize}
  \item ValueKey -- to distinguish widgets based on specific values, as items in a list.

\begin{lstlisting}
items.map((item) => ListTile(
    title: Text(item.name),
    key: ValueKey<String>(item.id),
  );
\end{lstlisting}

  \item ObjectKey -- to work with intricate data structures, that helps identify widgets based on their underlying data.

\begin{lstlisting}
Map<String, dynamic> userData = {/* Complex data */}; 
return UserWidget(userData);
=> findWidgetByKey(ObjectKey(userData));
\end{lstlisting}

  \item UniqueKey -- to ensure that no two widgets share the same key by generating a unique key on each call.
  
\begin{lstlisting}
List<Widget> uniqueWidgets = [
  Widget(key: UniqueKey()),
  Widget(key: UniqueKey()),
  Widget(key: UniqueKey()),
];
\end{lstlisting}

  \item GlobalKey -- to access a widget from any part of the application.

\begin{lstlisting}
final myWidgetKey = GlobalKey<MyWidgetState>();
  // In your widget tree
  MyWidget(
    key: myWidgetKey,
  );
// Somewhere else in your app
MyWidgetState myWidget = myWidgetKey.currentState;
myWidget.doSomething();
\end{lstlisting}

\end{itemize}

\noindent But most of the time it's enough to use just a \q{super.key}-declaration by inheriting from any of core 
widgets. This approach is useful in scenarios where the key of the parent widget already serves the identification and 
differentiation purposes. It promotes a code reusability and keeps the codebase cleaner and better maintainable.


\newpage
\subsubsection{Processing in Parallel}

In Flutter, we have two powerful methods for managing parallel processing: \q{Isolate} and \q{Compute}. These 
mechanisms are essential for efficiently executing tasks concurrently, enhancing the performance and responsiveness. 
Isolates are like separate threads that run in parallel, while Compute simplifies parallelism by offloading 
computations to a background isolate, allowing the main thread to remain responsive to user interactions.

\q{Isolate}s are not limited to just one instance; we may create and oversee multiple of them concurrently. This 
flexibility allows to distribute workloads efficiently and utilize the full potential of parallel processing:

\begin{lstlisting}
class ClientData { /* ... */ }

void main() async {
  List<ClientData> clients = [
    ClientData('John', 35000), 
    // ... others
  ];
  final receiver = ReceivePort();
  // Spawn a new isolate for each client
  for (var client in clients) {
    await Isolate.spawn(_runIsolate,
        (client: client, port: receiver.sendPort));
  }
  receiver.listen(print);
}

void _runIsolate(dynamic message) {
  // ... a financial calculation, e.g., tax projection
  double taxProjection = calculateTax(message.client);
  message.port.send(taxProjection);
}
\end{lstlisting}

\noindent \q{Compute} streamlines the process by automatically setting up an isolate for parallel execution of the 
given function, reducing the need for extensive communication with the main isolate:

\begin{lstlisting}
final storage = await compute(_processTransactions, filePath);
\end{lstlisting}

\noindent The main difference between \q{Isolate} and \q{Compute} is in control and complexity. While Compute suits 
for straightforward tasks without intermediate communication; Isolate offers fine-grained control, ideal for complex 
tasks with extensive communication between isolates or those requiring multiple isolates. As a tip, we may employ the 
\q{Future.wait}-method to execute a list of \q{Future}-objects concurrently:

\begin{lstlisting}
final results = await Future.wait([dataFromApi(), dataFromFile()]);
\end{lstlisting}


\newpage
\subsubsection{Managing SDK Versions} \label{sdk}

Flutter Version Manager (FVM) streamlines the process of managing diverse Flutter versions across multiple projects, 
including the task of downgrading to an earlier SDK version. It serves as a command-line interface that efficiently 
handles multiple Flutter SDK versions:

\begin{lstlisting}[language=terminal]
# Install FVM 
$ dart pub global activate fvm

# Set Flutter SDK version
$ fvm use {version}

# Show all installed versions:
$ fvm list

# List of all available releases: 
$ fvm releases
\end{lstlisting}

\noindent Once the \q{fvm use}-command is executed, it generates a \q{.fvm}-directory within the project. This directory 
contains a relative symlink connected to the cache of the chosen Flutter SDK version. Therefore, every command needs to 
be prefixed with \q{fvm}:

\begin{lstlisting}[language=terminal]
$ fvm flutter {command}
$ fvm dart {command}
\end{lstlisting}

\noindent In addition, it's widely used "flavors". It's a set of configurations that allow to build different versions
of the same app, each with different settings, and behavior. Flavors enable to have multiple versions of an app within 
a single codebase. Each flavor can have its own set of resources, API endpoints, or specific configurations. This is 
helpful in scenarios where the same app is targeted for different environments but needs to maintain individual 
characteristics for each:

\begin{lstlisting}[language=terminal]
# Install Flavor
$ fvm use {version} --flavor {flavor_name}

# Activate Flavor
$ fvm flavor {flavor_name}
\end{lstlisting}

\noindent By using flavors, we may streamline the process of building, testing, and deploying different versions of 
our application without having to manage separate codebases or repositories for each environment.


\newpage
\subsubsection{Debugging User Interface}

The \q{SemanticsDebugger}-widget provides a way to visualize and debug the semantics tree (\cref{img:bc-widget})
in your app (\cref{img:bc-widget-off}). It overlays a graphical representation of the semantics tree on top of the UI, 
displaying rectangles and labels corresponding to semantic nodes. This helps identify which elements are accessible and 
what information they convey, highlighting key accessibility properties like labels, actions, and roles.

\img{bootcamp/debugging-widget-off}{Application visualization}{img:bc-widget-off}

\begin{lstlisting}
SemanticsDebugger(
  child: MyApp(),
);
\end{lstlisting}

\noindent This widget offers several key features:

\begin{itemize}
  \item Each semantic node is outlined with a colored box, indicating its area on the screen.
  \item Inside these boxes, text displays the node's labels, roles, and actions.
  \item The debugger visualizes how touch interactions, such as taps and swipes, correspond to these nodes.
  \item Different colors help distinguish overlapping or nested nodes, making the structure easier to understand.
\end{itemize}

\img{bootcamp/debugging-widget}{Application visualization with SemanticsDebugger widget}{img:bc-widget}


\newpage
\subsubsection{Asserting in Tests}

In case of any test (unit, widget, or integration) creating for Flutter applications, the ultimate objective is to 
confirm that the actual outcomes align with the anticipated results.

\begin{lstlisting}
// Find and check Widget by its text
expect(find.text('text'), findsNothing);
// ... by Widget's type
expect(find.byType(DemoPage), findsOneWidget);
expect(find.byType(Container), findsWidgets);
\end{lstlisting}

\noindent Flutter helps to automate accessibility checks (that can be extended by inheriting 
\q{AccessibilityGuideline}-class):

\begin{lstlisting}
testWidgets('Verify Guidelines', (tester) async {
  await tester.pumpWidget(MyApp()); // Initialize Widget
  // Verify a tappable area (min 44x44 for iOS, 48x48 - Android)
  await expectLater(tester, 
      meetsGuideline(iOSTapTargetGuideline));
  await expectLater(tester, 
      meetsGuideline(androidTapTargetGuideline));
  // WCAG text contrast requirements
  await expectLater(tester, 
      meetsGuideline(textContrastGuideline));
  // Enforces all navigation elements to have a label
  await expectLater(tester,
      meetsGuideline(labeledTapTargetGuideline));
}
\end{lstlisting}

\noindent In the context of a complex layout, we may employ an image as a means to verify the rendered output (as a 
warning, fonts generation is a platforms specific even with \q{FlutterTest}-font, that is used if font family isn't 
specified or not available; check \ref{golden-image} for details):

\begin{lstlisting}
testWidgets('Compare with Image', (tester) async {
  await tester.pumpWidget(widget);
  final check = find.byType(MyApp);
  // By using a stored image file
  await expectLater(check, matchesGoldenFile('result.png'));
  // Compare with generated image
  final recorder = PictureRecorder();
  final imageCanvas = Canvas(recorder);
  imageCanvas.drawCircle(Offset.zero, 20.0, Paint());
  Picture image = recorder.endRecording();
  await expectLater(check, matchesReferenceImage(image));
});
\end{lstlisting}

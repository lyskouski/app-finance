% Copyright 2023 The terCAD team. All rights reserved.
% Use of this content is governed by a CC BY-NC-ND 4.0 license that can be found in the LICENSE file.

\subsubsection{AWS Cloud}

We've previously discussed the capability to publish a web package to GitHub Pages (see \ref{deploy-web}), but the same process can be used to release our application on the cloud. Here is a brief example of Amazon Simple Storage Service (AWS S3) usage:

\begin{lstlisting}[language=yaml]
name: Deploy to AWS S3

on: 
  push:
    branches: [main]

jobs:
  deploy:
    runs-on: ubuntu-latest
    steps:
    - name: Checkout
      uses: actions/checkout@main

    - name: Configure AWS Credentials
      uses: aws-actions/configure-aws-credentials@v1
      with:
        aws-region: ${{ secrets.AWS_REGION }}
        aws-access-key-id: ${{ secrets.AWS_ACCESS_KEY_ID }}
        aws-secret-access-key: ${{ secrets.AWS_SECRET_ACCESS_KEY }}
      env:
        AWS_S3_BUCKET: ${{ secrets.AWS_S3_BUCKET }}

    - name: Compile
      run: flutter build -v web --release

    - name: Deploy
      run: aws s3 sync ./build/web s3://${{ secrets.AWS_S3_BUCKET }} --delete
\end{lstlisting}

\noindent Upon configuring a DNS record with the provided \q{AWS S3}-URL, it should resemble the following:

\begin{lstlisting}[language=terminal]
[Type]	 [Host]        [Value]
APEX	   tercad.pt.    d1eojshauhnf9p.cloudfront.net.
\end{lstlisting}

\noindent As a step back, this web application can be deployed in any AWS region (\cref{img:cloud-region}) that supports 
the services used by the application.

\img{cloud/region}{Select Region}{img:cloud-region}

In the AWS Management Console, under "Storage", click "Services", and then choose "S3", select "Create Bucket".
Enter a globally unique bucket name. From the drop-down list, select the region we plan to use. Click the "Create" 
button in the lower-left corner of the dialog box without selecting the bucket from which to copy the settings.

To send all files and subdirectories from a "local directory", we will use GitHub Actions, as shown at the beginning.

With bucket policies, we can specify who is allowed to access the content in our S3 buckets. Bucket policies are JSON 
documents that specify which principals are allowed to perform various actions on the objects in our bucket. The 
following example bucket policy allows public read access to all objects in the specified bucket:

\begin{lstlisting}[language=yaml]
{
  "Version": "2012-10-17",
  "Statement": [
    {
      "Effect": "Allow",
      "Principal": "*",
      "Action": "s3:GetObject",
      "Resource": "arn:aws:s3:::our-bucket-name/*"
    }
  ]
}
\end{lstlisting}

\noindent Replace \q{our-bucket-name} with the actual name of our S3 bucket. This policy allows anyone (denoted by 
\q{Principal: "*"} to perform the \q{s3:GetObject} action, which is necessary for reading objects from the bucket.

By default, objects in an S3 bucket are accessible via a URL with the structure http://.amazonaws.com//. To serve 
resources from the root URL (i.e., /index.html), we must enable the website hosting feature in our bucket. On the bucket 
details page in the S3 console, open the "Properties" tab. Select the "Static website hosting". Select "Use" this bucket 
to host a website and enter index.html in the root document field. Leave the other fields blank.
After that, our objects will be accessible at a website address that depends on the AWS region of the bucket: 
.s3-website-.amazonaws.com. For example, if our bucket is named \q{tercad-web} and is located in the US East (N. 
Virginia) region, the URL would be \q{http://tercad-web.s3-website-us-east-1.amazonaws.com}.

~

\noindent For more details check: https://docs.aws.amazon.com\\
\href{https://docs.aws.amazon.com/AmazonS3/latest/userguide/HostingWebsiteOnS3Setup.html}{/AmazonS3/latest/userguide/HostingWebsiteOnS3Setup.html}

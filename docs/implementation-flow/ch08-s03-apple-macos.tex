% Copyright 2023 The terCAD team. All rights reserved.
% Use of this content is governed by a CC BY-NC-ND 4.0 license that can be found in the LICENSE file.

\subsubsection{Apple Store - macOS}

The process of creating \q{macOS}-package for a distribution \issue{287}{} is almost the same as for \q{iOS} 
(\ref{d-ios}) with a few deviations.

As previously shown, the provisioning profile can be installed on the device as a whole, but the practice is to embed 
the profile within the application itself (iOS expects to find the profile at \q{MyApp.app/embedded.mobileprovision}, 
macOS expects to find the profile at \q{MyApp.app/Contents/embedded.provisionprofile}; \q{.provisionprofile}-file can 
be obtained from selecting Mac App Store in 
\href{https://developer.apple.com/account/resources/profiles/add}{https://developer.apple.com/account/resources/profiles/add}):

\begin{lstlisting}[language=yaml]
- name: Run macOS Build
  run: flutter build -v macos --release

- name: Add the Provisioning Profile
  env:
    PROVISIONING_MAC_BASE64: ${{ secrets.APPLE_MAC_PROFILE }}
  run: |
    PP_PATH=$RUNNER_TEMP/Fingrom_macOS.provisionprofile
    echo -n "$PROVISIONING_MAC_BASE64" | base64 --decode \
        --output $PP_PATH
    CONTENT=build/macos/Build/Products/Release/Fingrom.app/Contents
    cp $PP_PATH $CONTENT/embedded.provisionprofile
\end{lstlisting}


\noindent As a note, the distribution certificate "3rd Party Mac Developer Application" comes from selecting Mac App 
Distribution in
\href{https://developer.apple.com/account/resources/certificates/add}{https://developer.apple.com/account/resources/certificates/add}.
The certificate "3rd Party Mac Developer Installer" can be taken by selecting Mac Installer Distribution from the same 
page. And the application has to signed by using both of them:

\begin{lstlisting}[language=yaml]
- name: Import macOS Distribution Certificate
  if: matrix.target == 'macOS'
  uses: apple-actions/import-codesign-certs@v2
  with: 
    keychain: signing_distr
    p12-file-base64: ${{ secrets.APPLE_MAC_P12 }}
    p12-password: ${{ secrets.APPLE_P12_KEY }}

- name: Sign macOS Package
  if: matrix.target == 'macOS'
  run: |
    # Sign ID is taken from a certificate (Developer ID)
    codesign --deep --force -s "8PVKPQ758L" --entitlements \
      macos/Runner/Release.entitlements "Fingrom.app"
  # build_path: build/macos/Build/Products/Release
  working-directory: ${{ matrix.build_path }}

- name: Import macOS Install Certificate
  uses: apple-actions/import-codesign-certs@v2
  with: 
    keychain: signing_distr
    p12-file-base64: ${{ secrets.APPLE_MAC_INSTALLER }}
    p12-password: ${{ secrets.APPLE_P12_KEY }}

- name: Upload macOS Package to Apple Store
  run: |
    xcrun productbuild --component Fingrom.app \
        /Applications/ unsigned.pkg
    xcrun productsign --sign "8PVKPQ758L" unsigned.pkg Fingrom.pkg
    # Validate package
    xcrun altool --validate-app --type macos -f Fingrom.pkg \
      --apiKey ${{ secrets.APPLE_API_KEY }} \
      --apiIssuer ${{ secrets.APPLE_API_ISSUE }}
    # Upload package
    xcrun altool --upload-app --type macos -f Fingrom.pkg \
      --apiKey ${{ secrets.APPLE_API_KEY }} \
      --apiIssuer ${{ secrets.APPLE_API_ISSUE }}
  working-directory: ${{ matrix.build_path }}
\end{lstlisting}

\img{distributing/app-store-macos}{Apple Store Connect: macOS Release}{img:d-apple-macos}

\noindent The submission might fail because of the next error:

\begin{lstlisting}[language=terminal]
Error: Asset validation failed App sandbox not enabled. 
  The following executables must include the 
  "com.apple.security.app-sandbox" entitlement with a 
  Boolean value of true in the entitlements property list
\end{lstlisting}

\noindent While \q{app-sandbox} is enabled by default for both compilations (Debug and Release), the problem is in a 
missed \q{com.apple.security.inherit}. To enable sandbox inheritance, a child target must use exactly two App Sandbox 
entitlement keys:

\begin{lstlisting}[language=xml]
<!-- ./macos/Runner/Release.entitlements -->
<dict>
  <key>com.apple.security.app-sandbox</key>
  <true/>
\end{lstlisting}
{
\xpretocmd{\lstlisting}{\vspace{-14pt}}{}{}
\begin{lstlisting}[firstnumber=5, language=xml, backgroundcolor=\color{backgreen}]
(*@\kdiff{+}@*) <key>com.apple.security.inherit</key>
(*@\kdiff{+}@*) <true/>
\end{lstlisting}
}

\noindent It's important to note that the end-of-line (EOL) character for the \q{macos/Runner/Release.entitlements}-file 
should be set to "LF" (Line Feed). Using "LF" as the EOL character is a common practice in macOS development, and 
any other formatting might lead to a not recognized entitlements list.

% Copyright 2023 The terCAD team. All rights reserved.
% Use of this content is governed by a CC BY-NC-ND 4.0 license that can be found in the LICENSE file.

\newpage
\subsubsection{Microsoft Store}

To distribute a Windows app through the Microsoft Store, we should create a Microsoft Partner Center account
(\href{https://partner.microsoft.com}{https://partner.microsoft.com}). By going through the registration form, it 
would be needed to pay the initial fee (roughly 80 Euro with TAX). The registration would be a prerequisite for 
creating the distributable MSIX-build, since a few properties for the build should be aligned with the information on 
our Partner Center (\cref{img:d-windows}). 

The most straightforward method for generating an MSIX-installer for our project is by utilizing the 
\q{msix}-package \issue{209}{}:

\begin{lstlisting}[language=terminal]
$ flutter pub add --dev msix
\end{lstlisting}

\begin{lstlisting}[language=yaml]
# ./pubspec.yaml
msix_config:
  display_name: Fingrom MSIX
  publisher: CN=78144012-EC6A-4BE8-97BF-A392EF55482E
  publisher_display_name: terCAD
  identity_name: terCAD.FingromMSIX
  logo_path: web/icons/Icon-512.png
  capabilities: internetClient # permissions
  store: true # distributable package
\end{lstlisting}

\img{distributing/win-store}{Microsoft Store -- Product Identity}{img:d-windows}

\noindent As a tip, MS Store requires to use all zeros for the version definition in case of a manual upload:

\begin{lstlisting}[language=terminal]
[X] Package acceptance validation error: Apps are not allowed
 to have a Version with a revision number other than zero 
 specified in the app manifest.
\end{lstlisting}

\begin{lstlisting}[language=terminal]
# Prepare Windows release
$ flutter config --enable-windows-desktop
$ flutter build -v windows \
  --build-name=${{ version }} \
  --build-number=${{ build_number }} \
  --release --obfuscate \
  --split-debug-info=build/windows/symbols

# Generate the MSIX-installer
$ dart run msix:create \
  --version 0.0.0.0	\
  --output-path ${{ matrix.build_path }} \
  --build-windows false \
  --capabilities internetClient
\end{lstlisting}

\noindent MSIX installers should be signed with a certificate to verify that apps are installed and updated from trusted 
sources. However, if we publish to the Microsoft Store, the store will automatically sign the installation package. 
Additionally, we may use App Center (\href{https://appcenter.ms}{https://appcenter.ms}) as an early access platform for 
users. It contains Analytics and Crash Reports that can be used across multiple platforms.

To streamline the distribution process, we might use MS Store CLI (command line interface), a tool designed to 
interface with the Microsoft Store submission API for publishing Windows apps to the Microsoft Store. To get started 
with this integration, we'll need to establish a connection between our Microsoft Partner Center account and an Azure 
AD application. We have to retrieve essential information like Tenant Name, Tenant ID, Client ID, and Client Secret. 
We would start from creating a new tenant at 
\href{https://partner.microsoft.com/en-us/dashboard/account/v3/tenantmanagement}{partner.microsoft.com/dashboard/account/v3/tenantmanagement}
page by clicking on "Create", and completing the necessary information (\q{Tenant ID}). Next, we'll access our Azure AD 
account using the new tenant credentials via \href{https://entra.microsoft.com}{https://entra.microsoft.com}. Within 
Azure AD, by navigating to "Applications", we'll proceed with a new registration (as a note, the "Redirect URI"-field 
might be blank). Under the "Certificates \& secrets" tab, we can generate a new certificate. This certificate will be 
utilized by our CI/CD pipeline, especially "Value"-field within this certificate represents our \q{Client Secret} for 
the CLI. To finalize the configuration, we'll visit the User Management 
(\href{https://partner.microsoft.com/en-us/dashboard/account/v3/usermanagement}{https://partner.microsoft.com/en-us/dashboard/account/v3/usermanagement})
to associate the created "MS Entra application", and grant a "Development" access to enable package uploads 
(\cref{img:d-win-store}).

\begin{lstlisting}[language=terminal]
# Tenant ID  
https://partner.microsoft.com/en-us/dashboard/account/v3/tenantmanagement
# CLient ID & Client Secret
https://partner.microsoft.com/en-us/dashboard/account/v3/usermanagement
# Seller ID 
https://partner.microsoft.com/en-us/dashboard/account/v3/organization/legalinfo
\end{lstlisting}

\img{distributing/win-store-key}{Microsoft Partner Center: Credentials for MS Store CLI}{img:d-win-store}

\noindent With all of that credentials taken we may activate the distribution pipeline:

\begin{lstlisting}[language=yaml]
- name: Setup MS Store CLI
  uses: microsoft/setup-msstore-cli@v1
- name: Publish MSIX to the Microsoft Store
  run: |
    msstore reconfigure --tenantId ${{ secrets.WINDOWS_TENANT_ID }} --clientId ${{ secrets.WINDOWS_CLIENT_ID }} --clientSecret ${{ secrets.WINDOWS_SECRET }} --sellerId ${{ secrets.WINDOWS_SELLER_ID }}
    msstore publish -v "${{ matrix.build_path }}"
\end{lstlisting}

\noindent Important to notice that it should not be the first submission of an application by using \q{msstore}, 
otherwise the build will fail with an error message:

\begin{lstlisting}[language=terminal]
Unhandled exception: System.NotSupportedException: 
  Cannot show selection prompt since the current 
  terminal isn`t interactive.
\end{lstlisting}

\noindent To address this error, we'll need to install \q{msstore} on our local machine and complete the necessary 
prompts for the same scope of commands. The first semiautomated submission is done (\cref{img:d-win-upload}):

\begin{lstlisting}[language=terminal]
# Connect
$ msstore reconfigure --tenantId ...
Configuration saved!
Awesome! It seems to be working!
# Init to set "msstore_appId" for the config
$ msstore init --publisherDisplayName terCAD
Ok! Found your apps!                    
This seems to be a Flutter project.
Let`s set it up for you!
# Send to Microsoft Partner Center
$ msstore publish -v
Please enter the following information: ... # Category, Languages
Uploading Bundle to Azure blob --------- 100%
\end{lstlisting}

\img{distributing/win-upload}{Microsoft Partner Center: Results of submission}{img:d-win-upload}

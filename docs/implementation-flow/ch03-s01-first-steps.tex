% Copyright 2023 The terCAD team. All rights reserved.
% Use of this content is governed by a CC BY-NC-ND 4.0 license that can be found in the LICENSE file.

\subsection{Configuring Environment} \label{first-step}
\markboth{Prototyping}{Configuring Environment}

Let's assume we are starting from scratch, knowing nothing neither about Flutter, nor Dart. So, the better way to start 
from that position is to use some templates (and Flutter kindly provides us such an option).

But, initially, we do need to download and install Flutter (\href{https://flutter.dev}{https://flutter.dev}) by 
following the installation instructions specific to operating system that is used (as an example, for Windows it 
would be needed to register a library directory location in the system's \q{PATH}-variable after downloading its 
sources, \cref{img:fs-windows-path}).

\img{first-steps/windows-path-variable}{Windows 10 \q{PATH}-variable location}{img:fs-windows-path}

The choice of a better Integrated Development Environment (IDE) for the Flutter development process is something 
negligible, widely proposed Visual Studio Code, Android Studio, IntelliJ IDEA; but the truth is that nobody cares, 
what was used (either VIM, Notepad, or nano), if the work is done. Nonetheless, further discussions in the book would 
be relevant to Visual Studio Code (\href{https://code.visualstudio.com/}{https://code.visualstudio.com/}) usage in 
"how-to configure IDE" examples if nobody minds.

\noindent Let's start from an Extensions Marketplace usage for Visual Studio Code by installing a few helpful extensions:

\begin{lstlisting}[language=terminal]
$ code --install-extension Dart-Code.dart-code
$ code --install-extension Dart-Code.flutter
# To display a summary for a tests execution with coverage
$ code --install-extension flutterando.flutter-coverage
# To display a color icon on front of its definition
$ code --install-extension circlecodesolution.ccs-flutter-color
# To simplify a work with Git-repository
$ code --install-extension eamodio.gitlens
# To check misprints
$ code --install-extension streetsidesoftware.code-spell-checker
\end{lstlisting}

\noindent and proceed with configuring project settings to apply \q{dart}-formatting on \q{Save}-action, display 
localization files in a \q{JSON}-language mode, and show the relations of a nested widget structure:

\begin{lstlisting}[language=bash]
# ./.vscode/settings.json 
{
    "dart.lineLength": 120, # Dart-formatter works with 80 symbols
    "editor.rulers": [120], # $ dart format -l 120
    "[dart]": {                # ./analysis_options.yaml > rules
        "editor.rulers": [120],# lines_longer_than_80_chars: false
        "editor.formatOnSave": true,
    },
    "files.associations": {
        "*.arb": "json"
    },
    "dart.previewFlutterUiGuidesCustomTracking": true,
    "dart.previewFlutterUiGuides": true,
}
\end{lstlisting}

\noindent Right after that IDE will ask regarding the Flutter Software Development Kit (SDK) folder (it's the directory 
where the Flutter was installed / downloaded into). Inside that folder, we may find a various directories and files, 
including:
\q{bin} -- Flutter command-line tools, such as \q{flutter}, \q{dart}, and other utilities;
\q{cache} -- cached files and downloaded packages;
\q{doc} -- documentation;
\q{examples} -- code examples and sample projects;
\q{packages} -- core Flutter packages and dependencies.

After those steps, from a Command Palette (by pressing \key{Ctrl} + \key{Shift} + \key{P}, or \key{Cmd} + \key{Shift} + 
\key{P} on macOS) will become available "Flutter: New Project"-command, by clicking on which prompt will ask regarding
project name and location to generate a basic structure. Going farther, let's open the project in IDE and frustrate a 
minute about a wide variety of different scaring files that's been generated. 

\newpage
\noindent Breathe... we do need only a one of them by now that's located in the \q{lib}-directory. The 
\q{main.dart}-file is the project's central orbit, representing the entry point of an application. Here is our first 
code that's been generated by the \q{flutter create}-command:

\begin{lstlisting}
// ./lib/main.dart
import 'package:flutter/material.dart';
// Entry point
void main() {
  runApp(const MyApp());
}
// 'Stateless' means missing any saved states of MyApp
class MyApp extends StatelessWidget {
  // 'super' - parent class, 'key' - widget unique identifier
  const MyApp({super.key}); // Class constructor
  
  // '@override' - replace implementation from its parent class
  @override
  Widget build(BuildContext context) { // Returns 'Widget' instance
    return MaterialApp(
      title: 'Flutter Demo',
      // Application theme definition
      theme: ThemeData(
        // Basic color scheme
        colorScheme: ColorScheme.fromSeed(
          seedColor: Colors.deepPurple,
        ),
        // Configuration of Widgets' styles
        useMaterial3: true,
      ),
      // Main page definition
      home: const MyHomePage(title: 'Flutter Demo Home Page'),
    );
  }
}
// 'Stateful' to re-render widget by changed internal states
class MyHomePage extends StatefulWidget {
  // To mark 'title' as a mandatory for constructor [line 26]
  const MyHomePage({super.key, required this.title});
  // 'final' - cannot be changed afterwards
  final String title;
  // Wrap an object for a state management
  @override
  _MyHomePageState createState() => _MyHomePageState();
}

// Underscore is used for private classes, methods, and variables
class _MyHomePageState extends State<MyHomePage> {
  int _counter = 0;
  // Internal method to update '_counter' state
  void _incrementCounter() {
    // 'setState' - a special method to save state
    // ... and re-trigger 'build'-method
    setState(() {
      _counter++;
    });
  }
  // Mandatory method for any Widget-based classes
  @override
  Widget build(BuildContext context) {
    return Scaffold(
      // Pinned to the top container
      appBar: AppBar(
        backgroundColor:
            Theme.of(context).colorScheme.inversePrimary,
        title: Text(widget.title),
      ),
      body: Center(
        child: Column(
          mainAxisAlignment: MainAxisAlignment.center,
          children: <Widget>[
          // Usage of 'const' is recommended to improve performance
          // ... for any Widget without dynamic parameters
            const Text(
              'You have pushed the button this many times:',
            ),
            Text(
              '$_counter', // '$' - for a variable interpolation
              style: Theme.of(context).textTheme.headlineMedium,
            ),
          ],
        ),
      ),
      floatingActionButton: FloatingActionButton(
        onPressed: _incrementCounter,
        tooltip: 'Increment',
        child: const Icon(Icons.add),
      ),
    );
  }
}
\end{lstlisting}

\noindent The results of a compilation (\cref{img:fs-app}) can be viewed by triggering \q{flutter run}-command from a 
terminal (as an example, for Visual Studio Code by clicking on "View" from the top menu, selecting "Terminal"; short 
key \key{Ctrl} + \key{backtick}). With an inactive emulator state and a variety of options, it may ask to clarify 
the type of execution:

\begin{lstlisting}[language=terminal]
$ flutter run
Connected devices:
Linux (desktop) * linux  * linux-x64      * KDE neon 5.27 ...
Chrome (web)    * chrome * web-javascript * Chromium 118.0...
[1]: Linux (desktop)
[2]: Chrome (web)
Please choose one (or "q" to quit): 
\end{lstlisting}

\img{first-steps/app-template}{First run with autogenerated application}{img:fs-app}

\noindent A valuable tip for any development process is to activate a debug mode, that can be done simply by adding 
\q{--debug} to the previous console command (but the option can be omitted since it's used by default). Then, 
the process is similar to any other programming language: set breakpoints in a code by clicking in IDE on the left 
margin of the desired line, use the command, inspect variables and their states by stepping through the code. Instead 
of any debug mode it can be simply used \q{print}-function to plot anything into console. By taking into account that 
we might know exactly nothing about Flutter (and, possibly, any other programming languages), let's return to that 
statement when our application (and we) will become more mature... too early without any business logic on board.

What we can do in addition, is to emulate a real device usage. That's achievable with the Android SDK by downloading 
(\href{https://developer.android.com/studio}{https://developer.android.com/studio}), installing, and going through 
the initial wizard setup (nothing special by now, choose everything by default). One additional step is to create a 
virtual device (\cref{img:fs-android})... a couple of clicks and we're done (\cref{img:fs-create}).

Once we've set up Android SDK and triggered "Start" for the emulator, Flutter will recognize its availability from
an execution \q{flutter run}. Flutter app will be build and launched on the connected Android emulator (or even real 
device, connected via USB; check manuals if needed).\\

\img{first-steps/android-studio}{Android Studio -- Starting Page}{img:fs-android}
\img{first-steps/android-studio-create}{Android Studio  -- Emuator Creator}{img:fs-create}

\noindent Let's dive into the coding!

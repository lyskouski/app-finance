% Copyright 2023 The terCAD team. All rights reserved.
% Use of this content is governed by a CC BY-NC-ND 4.0 license that can be found in the LICENSE file.

\subsection{Using `TextEditingController` for tabbing}
\markboth{Optimizing UI/UX Flow}{Using `TextEditingController` for tabbing}

\subsubsection{Reasoning}

Transition across fields on a form (after pressing the enter key) eliminates the need for users to manually 
click or tap on the desired field, resulting in a smoother data entry process. Users can swiftly enter data in a 
sequential manner, mimicking the natural flow of their thoughts and reducing repetitive actions. This helps prevent 
errors and ensures that all necessary information is captured. Users can confidently complete forms, knowing that 
they are guided through the process and not likely to overlook any fields, resulting in a more error-free experience.

Autofocus and the enter key functionality enhance the accessibility of the application for individuals who rely on 
keyboard navigation or assistive technologies. Users who prefer or require keyboard-based interactions can easily 
navigate through the application and complete forms without the need for mouse or touch input. This inclusivity ensures 
that the application is accessible to a wider range of users, including those with motor impairments or visual 
challenges.

By implementing these features, the application maintains consistency with user expectations, reducing the learning 
curve, ensuring a familiar and intuitive experience, and allure breathe life into the application's usability,
ushering users into a realm where efficiency reigns supreme.

\subsubsection{Relevant Changes}

\href{https://github.com/lyskouski/app-finance/issues/46}{https://github.com/lyskouski/app-finance/issues/46}

\subsection{Technical Details}

For a single field it's just `autofocus: true`-attribute on `TextFormField`-widget while when we do want to control 
further steps autofocus, a basic solution would be the following:

\begin{lstlisting}
class MyForm extends StatefulWidget {
  @override
  _MyFormState createState() => _MyFormState();
}

class _MyFormState extends State<MyForm> {
  late FocusNode _focusNode1;
  late FocusNode _focusNode2;
  late TextEditingController _controller1;
  late TextEditingController _controller2;

  @override
  void initState() {
    super.initState();
    _focusNode1 = FocusNode();
    _focusNode2 = FocusNode();
    _controller1 = TextEditingController();
    _controller2 = TextEditingController();
  }

  @override
  void dispose() {
    _focusNode1.dispose();
    _focusNode2.dispose();
    _controller1.dispose();
    _controller2.dispose();
    super.dispose();
  }

  @override
  Widget build(BuildContext context) {
    return Scaffold(
      appBar: AppBar(title: Text('Form Example')),
      body: Padding(
        padding: EdgeInsets.all(16.0),
        child: Column(
          children: [
            TextField(
              focusNode: _focusNode1,
              controller: _controller1,
              autofocus: true,
              decoration: InputDecoration(labelText: 'Field 1'),
              textInputAction: TextInputAction.next,
              onEditingComplete: () {
                FocusScope.of(context).requestFocus(_focusNode2);
              },
            ),
            TextField(
              focusNode: _focusNode2,
              controller: _controller2,
              decoration: InputDecoration(labelText: 'Field 2'),
              textInputAction: TextInputAction.done,
              onEditingComplete: () {
                // Perform any final actions here
                _submitForm();
              },
            ),
          ],
        ),
      ),
    );
  }

  void _submitForm() {
    // Handle form submission
    // e.g., validate input, perform data processing, etc.
    print('Field 1 value: ${_controller1.text}');
    print('Field 2 value: ${_controller2.text}');
  }
}
\end{lstlisting}

\subsection{Implementation Notes}

[TBD]

If we'll implement as is, focus event will concurrent with an update request, that can be resolved by using a delay.

\begin{lstlisting}
// [OR] Future.delayed(Duration(milliseconds: 300), () {
Future.microtask(() {
  FocusScope.of(context).requestFocus(secondFocusNode);
});
\end{lstlisting}

Furthermore, we're not able to open our custom selectors anymore since `FocusScope.of(context).requestFocus` leads to 
re-build by any changed focus action. The problem is relevant to fact that we've been using `BuildContext` of our Form
whereas it should be taken context, where `DropdownButton`-element is located 

\begin{lstlisting}
class ListAccountSelector<T extends ListAccountSelectorItem> extends StatelessWidget {
  @override
  Widget build(context) {
    FocusController.setContext(context); // (!) fix
    // ... other code
\end{lstlisting}
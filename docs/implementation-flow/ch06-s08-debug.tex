% Copyright 2023 The terCAD team. All rights reserved.
% Use of this content is governed by a CC BY-NC-ND 4.0 license that can be found in the LICENSE file.

\subsection{Debugging Issue}
\markboth{Optimizing}{Debugging Issue}

Any application may encounter bugs and issues during after being released that might be hardly understandable 
(\cref{img:o-error}, \issue{421}{}).

\img{features/crash-stacktrace}{Firebase Chrashlytics: Failure report}{img:o-error}

\noindent One of the simplest yet effective methods for debugging in Flutter is logging. By strategically placing logger 
statements in our code, we may trace the flow of execution and monitor variable changes. For the case of distributed 
applications, it should be used loggers with external (to application) storage, as Sentry
(\href{https://pub.dev/packages/sentry\_flutter}{https://pub.dev/packages/sentry\_flutter}):

\begin{lstlisting}
// Initialization
runZonedGuarded(() async {
  Future<void> main() async {
    await SentryFlutter.init(
      (o) => o.dsn = 'https://{NAME}@sentry.io/{DSN}'
    );
  }
  runApp(MyApp());
}, (ex, trace) async {
  await Sentry.captureException(ex, stackTrace: trace);
});
// ... in the code
Sentry.captureMessage(
  'A message with some info',
  withScope: (scope) {
    scope.setTag('variable', 'state');
  },
);
\end{lstlisting}

\noindent Once the issue has been successfully replicated locally, we might place breakpoints in our code to halt 
execution at specific points. By default, when executing the \q{flutter run}-command, the application runs in a
debug mode. This enables us to scrutinize variables and navigate through the code step by step, facilitating a more 
granular inspection of the codebase. But let's avoid that, since the debugging process (either local or production) 
should have a unified approach -- logs.

In the case of utilizing a web distribution, it becomes necessary to find a means to examine Web Progressive Applications 
(WPA). The majority of mobile browsers facilitate a remote debugging through their developer tools. For instance, 
in Chrome, we may use \href{chrome://inspect\#devices}{chrome://inspect\#devices} to enable access to a phone that is 
connected via a USB. Underneath it would be shown a list of recognized devices (\cref{img:o-devtool}). By clicking on 
the respective device, the inspect panel is activated, enabling the detailed examination and debugging.

\img{features/devtool-devices}{Google Chrome: Devices Inspector}{img:o-devtool}

\noindent The phone might not be recognized after being connected. In that case we should turn on the "USB debugging".
As for Android, those options are available on \q{Developer options} from \q{Settings}-menu after clicking 7 times 
on \q{Build Number} (from "About phone"-submenu).

For the case of "\emph{Pending authentication: please accept debugging session on the device}", consider experimenting 
with various USB cables or connecting to different ports on a machine. Additionally, we should attempt a sequence of 
command prompts to manage the Android Debug Bridge (ADB): start from \q{adb kill-server}, followed by 
\q{adb start-server}, and finally \q{adb devices} to verify and check the status of connections, and retick the 
"Discover network targets"-checkbox. This sequence might help resolve authentication-related issues during the debugging 
session.

An additional tip is to modify the USB connection type to \q{MIDI} instead of file transfers. Simply slide down the 
main header to access the configuration settings and make the necessary adjustment.

~

\noindent All of those things were needed to transform our initial error (\cref{img:o-error}) into a more clear trace 
(\cref{img:o-error-details}, 
\href{https://github.com/flutter/flutter/issues/138880}{https://github.com/flutter/flutter/issues/138880}):

\img{features/crash-stacktrace-details}{Google Chrome: Failure detection}{img:o-error-details}

\noindent The issue at hand is exclusive to Progressive Web App (PWA) development, particularly linked to transliteration 
corruption. This problem emerges \emph{(in Flutter 3.16.0 with Dart 3.2.0)} due to the two-step compilation process 
that Dart code undergoes. Initially, it is compiled into optimized JavaScript for deployment, and then this JavaScript 
undergoes Just-In-Time (JIT) compilation by the web platform into a native code during runtime. The intricacies of this 
compilation pipeline introduce a potential source of errors, as observed in the described scenario.

Previously, there was a brief mention of the option to build with Web Assembly (\ref{deploy-web}). Web Assembly is a 
low-level format that resides closer to the abstraction level of machine code compared to JavaScript. This proximity 
results in enhanced startup times and a more predictable execution flow. Opting for Web Assembly eliminates the need 
for the two-step compilation process, making it a potential solution amidst various uncertainties. However, it's worth 
noting that Web Assembly, introduced in 2020, remains experimental and introduces its set of considerations and 
potential trade-offs.

Occasionally, when an urgent solution is required, the quickest remedy may involve reverting \issue{424}{} the made 
changes. In our context, this entails downgrading Flutter to a previous version (that can be retrieved from CI/CD). 
The rollback can be achieved through FVM (\ref{sdk}) or directly by following a step-by-step process:

\begin{lstlisting}[language=terminal]
# Check the current state
$ flutter doctor
[V] Flutter (Channel stable, 3.16.0, on ...)
# Identify the SDK location
$ which flutter
/home/Programs/flutter/flutter/bin//flutter
$ cd /home/Programs/flutter/flutter/bin
$ git fetch --tags
$ git checkout 3.13.9
$ flutter doctor
# Update metadata
$ cd ~/project
$ flutter create .
\end{lstlisting}

\noindent And from here on out, we need to nail down the SDK version in all of our CI/CD processes:

\begin{lstlisting}[language=yaml]
- uses: subosito/flutter-action@v2
  with:
\end{lstlisting}
{
\xpretocmd{\lstlisting}{\vspace{-12pt}}{}{}
\begin{lstlisting}[firstnumber=2, backgroundcolor=\color{backred}]
(*@\kdiff{-}@*)      channel: 'stable'
\end{lstlisting}
\begin{lstlisting}[firstnumber=2, backgroundcolor=\color{backgreen}]
(*@\kdiff{+}@*)      flutter-version: 3.13.9
\end{lstlisting}
\begin{lstlisting}
    cache: true
\end{lstlisting}
}

\noindent To restore back the state:

\begin{lstlisting}[language=terminal]
$ flutter channel stable
$ flutter upgrade
\end{lstlisting}

\noindent However, addressing the original issue remains elusive as it extends beyond our sphere of influence. Even if 
we will revert almost all the changes made between two releases of the application... even by going further to primordial, 
the error would persist. This is because we have the control over the initial compilation of the web application but lack 
control over the Just-In-Time (JIT) transformation that occurs on a mobile device as a Progressive Web App (PWA).

Interestingly, this challenge led us to an unmanageable error state, necessitating the prompt action due to freezes 
associated with each opened selector (for everyone who is using the app as PWA). Throughout the re-implementation 
process of the component, we discovered an essential capability: the seamless transfer of data between pages.

\begin{lstlisting}
// ./lib/design/form/list_selector.dart
void onTap(BuildContext context) async {
  focusController.onFocus(this);
  final result = await Navigator.push(
    context,
    MaterialPageRoute(
      builder: (context) => ListSelectorPage(
        options: options,
        result: value,
        tooltip: tooltip,
        itemBuilder: getItemBuilder(),
      ),
    ),
  );
  widget.setState(result); (*@ \stopnumber @*)
}

// ./lib/design/form/list_selector_page.dart
onPressed: () => nav.pop<T?>(result as T?),
\end{lstlisting}

\noindent We have successfully extended our capabilities by not only transmitting a raw data, but also encapsulating a 
representation state within a function. Moreover, we have enhanced our component by introducing a clear-state button 
and adding functionality to automatically select the first element by pressing Enter \issue{424}{}. We can proudly 
commend ourselves for navigating challenges and enhancing our systems, much like adept problem solvers.

From Web Development I've taken a practice to revise myself on the weakest browser (that generates the highest variety
of errors). And for the Flutter it's a Progressive Web App (PWA) application usage.

As a note, we have refrained from utilizing breakpoints to debug our application step by step through the IDE. 
Let's avoid the vicious practices of modern programming that leads to inadvertent complications in the realm of 
concurrency and parallelism, where a breakpoint might influence the app's flow in a manner that could lead to even 
greater confusion. And, when we reach the point of feeling unable to comprehend the code, the problem might not be 
there.

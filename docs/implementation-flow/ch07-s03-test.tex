% Copyright 2023 The terCAD team. All rights reserved.
% Use of this content is governed by a CC BY-NC-ND 4.0 license that can be found in the LICENSE file.

\subsection{Researching}
\markboth{Productionizing}{Researching}

Marketing research comprises a range of methodologies \cite{Este21} employed to collect data and gain deeper insights 
into a company's target market. This information can be used for various purposes, including enhancing product 
development, improving user experiences, and formulating marketing strategies aimed at attracting high-quality prospects 
and increasing conversion rates.

While analytics (\ref{telemetry}) provides information on "what", research delves into the "why". Research is key to 
understanding the motivations and rationale behind users' actions (see \ref{usability}). For example, forensic marketing 
is an investigative practice within marketing research that is useful for identifying deficiencies in marketing 
effectiveness concerning customer reach. It can also be used to address marketing strategy and product-related issues 
and to bridge gaps in adapting to shifts in the competitive landscape and environmental dynamics. This includes 
competitive intelligence and responses to competitors. Additionally, forensic marketing borrows applications from 
forensic economic investigations, particularly in cases involving financial damages related to disputes over copyright 
infringement, product claims, or branding.

Furthermore, ethnography \cite{Mois06} offers methodological solutions for addressing theoretical questions central to 
cultural marketing and consumer research. It helps develop an understanding of cultures that may initially seem 
"foreign" or distant.

Both of the aforementioned practices, including marketing research in general, require a certain level of expertise. 
It's essential to start with a straightforward approach, such as scanning the market for similar solutions \issue{12}{} 
and identifying their distinctive features. The marketing research strategy should then involve data collection, 
surveys, focus groups, etc. Through thorough market research and open communication with users, the application can 
evolve to meet their needs and expectations, ultimately resulting in a more successful product.

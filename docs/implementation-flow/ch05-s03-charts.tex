% Copyright 2023 The terCAD team. All rights reserved.
% Use of this content is governed by a CC BY-NC-ND 4.0 license that can be found in the LICENSE file.

\subsection{Visualizing Data}
\markboth{Unleashing}{Visualizing Data}

With countless transactions, revenue streams, expenditures, and budgetary considerations, having a system that 
simplifies these complexities is essential. Traditionally, spreadsheets and numerical reports have been used to handle 
this complexity, but they often result in an overwhelming experience, especially as data volumes continue to grow.  This 
challenge requires solutions that effectively visualize and interpret data. Our financial accounting application employs 
data visualization for various use cases:

\begin{itemize}
  \item Cash Flow Analysis -- predicts financial trends, manage liquidity, and ensure operational continuity.

  \item Budget Monitoring -- helps to stay on track and avoid overspending.

  \item Financial Reporting -- provides summary reports, including annual reports, income statements, and balance 
  sheets, offering a snapshot of transactions within a defined period.

  \item Risk Management -- visualizes risk factors and their impact to make proactive decisions in mitigating financial 
  threats.

  \item Portfolio Management -- assess the performance of assets for informed investment decisions.
\end{itemize}

\noindent These categories encompass a wide range of tools and techniques. Initially, we will focus on delivering 
high-impact features quickly, since achieving the full scope may take months. As we move forward, we will progressively 
enhance the application by introducing a broader range of options.

To start, let's introduce the first chart for managing budget categories: The Forecast Chart. It provides a visual 
representation of historical data and uses mathematical models to forecast the future. The specific field and the type 
of data being analyzed determine these models. We'll start with the simplest one: Monte Carlo simulation, a 
computational technique that uses random sampling to approximate the behavior of complex systems or processes. This 
method is particularly valuable when dealing with systems involving many variables and intricate relationships. By 
simulating numerous scenarios through random sampling, Monte Carlo simulations provide a way to explore a wide range of 
possibilities and assess the likelihood of different financial scenarios:

\begin{lstlisting}
// ./lib/_classes/math/monte_carlo_simulation.dart
class MonteCarloSimulation {
  final Random rnd = Random();
  final int cycles;
  final double coercion;
  MonteCarloSimulation({this.cycles = 30, this.coercion = 1});
  List<Offset> generate(List<Offset> data, num step, num max) {
    final List<List<double>> distribution = [];
    // Loop through the scope (provided data points)
    for (int i = 0; i < data.length; i++) {
      final state = mcNormal(data[i].dy, coercion, cycles);
      // Loop through the states generated for each data point
      for (int j = 0; j < state.length; j++) {
        if (j >= distribution.length)
          distribution.add([]);
        distribution[j].add(state[j]);
      }
    }
    double posX = data.last.dx + step;
    List<Offset> result = [];
    int idx = 0;
    // Generate simulated data points for the forecast
    while (posX <= max) {
      result.add(Offset(
        posX, 
        distribution[idx][
          distribution[idx].length * rnd.nextDouble() ~/ 1],
          // where '~/ 1' is equal to '.toInt()'
      ));
      posX += step;
      idx++;
    }
    return result;
  }
  // To generate a list of random values using the simulation
  List<double> mcNormal(double mean, double stdDev, int samples) {
    List<double> results = [];
    // Perform the Monte Carlo simulation 
    // for the specified number of samples
    for (int i = 0; i < samples; i++) {
      results.add(_normalRandom(mean, stdDev));
    }
    return results;
  }
  // A random value based on a normal distribution
  double _normalRandom(double mean, double stdDev) {
    double u1 = rnd.nextDouble();
    double u2 = rnd.nextDouble();
    double z0 = sqrt(-2.0 * log(u1)) * cos(2 * pi * u2);
    return mean + stdDev * z0;
  }
}
\end{lstlisting}

\noindent It's left to generate the chart itself \issue{4}{}, which can be achieved using the \q{CustomPaint}-widget. 
The advantage of using this widget is that it allows us to separate the chart's line (the \q{painter}-property) from 
secondary data, such as the axes and background colors (the \q{foregroundPainter}-property):

\begin{lstlisting}
// ./lib/charts/forecast_chart.dart
final xMin = DateTime(now.year, now.month);
final xMax = DateTime(now.year, now.month + 1);
final bg = ForegroundChartPainter(
  yMin: 0.0,
  yMax: 140,
  xMin: xMin.millisecondsSinceEpoch.toDouble(),
  xMax: xMax.millisecondsSinceEpoch.toDouble(),
  yArea: [80, 120], // green area as a threshold
);
return SizedBox(
  height: size.height,
  width: size.width,
  child: CustomPaint(
    size: size,
    painter: ForecastChartPainter(
      data: data,
      yMax: yMax * bg.yMax / 100,
      xMin: xMin.millisecondsSinceEpoch.toDouble(),
      xMax: xMax.millisecondsSinceEpoch.toDouble(),
    ),
    foregroundPainter: bg,
    willChange: false, // to avoid re-build
    child: Padding(
      padding: EdgeInsets.only(top: indent / 4),
      child: Text(tooltip),
    ),
  ),
);
\end{lstlisting}

\noindent We created our own painters: \q{ForegroundChartPainter} (which will be used for all other charts), as well as 
the \q{ForecastChartPainter}. These classes are responsible for plotting graphical information on the \q{Canvas}:

\begin{lstlisting}
// ./lib/charts/painter/forecast_chart_painter.dart
class ForecastChartPainter extends CustomPainter {
  // ... properties declaration
  ForecastChartPainter({ /* ... */ });

  @override
  void paint(Canvas canvas, Size size) {
    for (final scope in data) {
      // Plot historical data
      _paint(canvas, scope.data, size, scope.color);
      final dx = scope.data.last.dx;
      final total = _sumY(scope.data);
      if (scope.data.length > 2 && dx < xMax && total < yMax) {
        final cycles = (xMax - dx) ~/ msDay;
        final forecast = [Offset(scope.data.last.dx, total)];
        forecast.addAll(MonteCarloSimulation(cycles: cycles).generate(scope.data, msDay, xMax - 2 * msDay));
        // Draw forecast line
        _paint(canvas, forecast, size, scope.color.withBlue(200).withValues(alpha: 0.4));
      }
    }
  }

// ... other stuff

  // Draw point to reflect data
  void _paintDot(Canvas canvas, Offset point, Color color) {
    final dot = Paint()..color = color;
    canvas.drawCircle(point, 2.2, dot);
  }
  // Draw a curve using four points
  _paintCurve(Canvas canvas, Offset p0, Offset k1, Offset k2, Offset p1, Color color) {
    final line = Paint()
      ..color = color
      ..style = PaintingStyle.stroke
      ..strokeWidth = 2;
    final path = Path()..moveTo(p0.dx, p0.dy);
    path.cubicTo(k1.dx, k1.dy, k2.dx, k2.dy, p1.dx, p1.dy);
    canvas.drawPath(path, line);
  }
}
\end{lstlisting}

\noindent That's a wrap! Following the implementation of various charts, including radial bar \issue{128}{}, column 
and bar race charts \issue{147}{}, OHLC (open, high, low, close) chart \issue{148}{}, gauge chart \issue{160}{} 
\issue{180}{}, pie chart \issue{179}{}, currency trades \issue{182}{};, we've curated a set of valuable metrics. 
These metrics (\cref{img:f-charts}) empower users to gain deeper insights into their financial situations.

\img{features/charts}{Visualization of Charts}{img:f-charts}

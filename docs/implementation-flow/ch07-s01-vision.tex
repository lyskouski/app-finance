% Copyright 2023 The terCAD team. All rights reserved.
% Use of this content is governed by a CC BY-NC-ND 4.0 license that can be found in the LICENSE file.

\subsection{Visioneering}
\markboth{Productionizing}{Visioneering}

Crafting a compelling product vision, the keystone of a product or service, demands a range of skills and discipline.
But most of the time it's started from basic questions \cite{Barr13}:

\begin{itemize}
  \item What is your vision for the enterprise?
  \item What is your purpose?
  \item Does a market exist and can you access it?
  \item Will you face competition?
  \item Can you clearly define your product and/or service?
  \item Have you formulated a plan?
  \item Do you need to involve other people?
  \item Do you need physical premises?
  \item How will you finance your enterprise?
  \item What potential risks may your enterprise face?
  \item What are your longer-term intentions and personal objectives?
\end{itemize}

\noindent Right now we've created a tiny application not even for managing personal finances, but for tracking expenses; 
but what if I tell you that the vision is \textbf{"to become a financial institution known for its integrity and ethical 
practices"} by declaring a core values as \q{Trust} (a solid reputation for reliability, transparency, and 
trustworthiness), \q{Community-Centric} (a sense of community among users, encouraging financial literacy), 
\q{Accessibility} (aspires to be universally accessible and serve users around the world, regardless of their 
capabilities and abilities). This vision reflects our unwavering commitment to providing financial services that go 
beyond mere functionality. We are dedicated to making a meaningful and positive impact on the financial well-being of
individuals and communities, promoting trust, inclusivity, and financial empowerment on a global scale with a motto: 
\textbf{"Financial Awareness is a Necessity"}.

We need to align our financial strategy with our vision of becoming a trustworthy financial institution. This involves 
tailoring our revenue, cost, and profit projections to match our unique objectives and the prevailing market conditions:

\begin{itemize}
  \item Revenue Projections: Start with conservative user growth estimates, recognizing that we're initially a small 
  application. Estimate the number of users who might opt for subscriptions.

  \item Cost Projections: Consider ongoing development and maintenance costs. As the user base expands, allocate 
  resources for customer support to handle inquiries and issues. Don't forget to budget for marketing and promotional 
  activities to acquire new users and retain existing ones. Given our commitment to integrity and ethical practices, 
  set aside funds for compliance and security measures to meet regulatory requirements and maintain user trust.

  \item Profitability: Understand that achieving profitability may take time, especially if we're transitioning from an 
  expense-tracking app to a financial institution.
\end{itemize}

[TBD] OKR and KPI

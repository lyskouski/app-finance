% Copyright 2023 The terCAD team. All rights reserved.
% Use of this content is governed by a CC BY-NC-ND 4.0 license that can be found in the LICENSE file.

\markboth{Productionizing}{Productionizing}

Transitioning from developing an application to deploying it in a production environment requires careful planning and 
consideration. It's valuable to ensure that the application is well-structured and follows best practices 
\ref{refactoring}. Implement a comprehensive testing strategy that includes unit, widget, and integration tests 
\ref{quality}, as well as a forecast \ref{benchmark} and a usability analysis \ref{usability}. Add logging and crash 
reporting tools to diagnose issues in a production environment \ref{telemetry}. Productionizing an application involves 
much more than writing code. It requires careful planning and attention to security and scalability, as well as a focus 
on user feedback and ongoing maintenance. 

Let's briefly address the most controversial topic, planning, as it becomes an issue from time to time (as in James O. 
McKinsey's publication regarding the productivity measurement of development teams, cited in \cite{McKi23}). We must 
commit to delivering the next scope of features to our stakeholders and customers as promised, while also maintaining a 
buffer to mitigate potential risks. Therefore, productivity measurement has become a key indicator for capacity 
allocation. The issue is that software engineering involves more than coding. It also involves making architectural 
decisions, conducting tests, performing security analyses, monitoring performance, and other valuable activities not 
mentioned. The key to success lies in ensuring transparency in both directions. Distrusting employees (e.g., fearing 
insider leaks) creates a cycle of mistrust that turns the development process into a black box. 

Instead of planning a specific N-week iteration or a broader quarterly increment, it's important to take a long-term 
perspective and make decisions that will last for decades. This approach emphasizes strategic thinking and creating 
solutions that will stand the test of time. It means considering how the software will evolve and adapt over the years, 
not just in the short term. This transformation shifts our focus from mere performance monitoring to stability 
measurement \cite{Heal23}, which may significantly enhance product delivery predictability. From this point forward, 
assess progress based not on the number of tasks completed, but on whether we accomplished our intended goals within the 
allocated time without succumbing to distractions \cite{Eyal20}.

The productionization process can be broken down into four interrelated actions: the Vision (Why?) \cite{Wall19}, crafting a Strategy (Where? How?) \cite{Lafl13}, setting clear Objectives (What?) \cite{Doer18}, and creating a Roadmap (So What? When?) \cite{Lomb17}.

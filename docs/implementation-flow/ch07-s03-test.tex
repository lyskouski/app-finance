% Copyright 2023 The terCAD team. All rights reserved.
% Use of this content is governed by a CC BY-NC-ND 4.0 license that can be found in the LICENSE file.

\subsection{Researching}
\markboth{Productionizing}{Researching}

Marketing research comprises a range of methodologies \cite{Este21} employed to collect data and gain deeper insights 
into a company's intended market. This gathered information serves various purposes, including the enhancement of 
product development, the improvement of user experiences, and the formulation of marketing strategies aimed at drawing 
in high-quality prospects and enhancing conversion rates.

While analytics (\ref{telemetry}) provides information on "what", research delves into the "why". The research
is the key to unraveling the motivations and rationale behind users' actions (\ref{usability}). For instance, forensic 
marketing, as an investigative practice within marketing research, is useful in identifying deficiencies in marketing 
effectiveness concerning customer reach, addressing marketing strategy and product-related issues, and bridging gaps in 
adapting to shifts in the competitive landscape and environmental dynamics. This includes competitive intelligence and 
responses to competitors. Additionally, forensic marketing borrows certain applications from forensic economics 
investigations, especially in cases involving financial damages related to disputes over copyright infringement and 
product claims or branding.

Furthermore, ethnography \cite{Mois06} offers methodological solutions for addressing the theoretical questions central 
to cultural marketing and consumer research. It aids in developing an understanding of cultures that may initially 
seem "foreign" or distant.

Both of the mentioned practices, including marketing research in general, require a degree of expertise. It's essential 
to start with a straightforward approach, such as scanning the market for similar solutions \issue{12}{} and identifying 
their distinctive features. Going further, the marketing research strategy should involve data collection, surveys, 
focus groups, and more. By conducting thorough market research and maintaining open channels of communication with our 
users, the application can evolve to meet their needs and expectations, ultimately leading to a more successful product.

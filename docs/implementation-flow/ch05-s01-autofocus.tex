% Copyright 2023 The terCAD team. All rights reserved.
% Use of this content is governed by a CC BY-NC-ND 4.0 license that can be found in the LICENSE file.

\subsection{Using `TextEditingController` for tabbing}
\markboth{Optimizing UI/UX Flow}{Using `TextEditingController` for tabbing}

\subsubsection{Reasoning}

Transition across fields on a form (after pressing the enter key) eliminates the need for users to manually 
click or tap on the desired field, resulting in a smoother data entry process. Users can swiftly enter data in a 
sequential manner, mimicking the natural flow of their thoughts and reducing repetitive actions. This helps prevent 
errors and ensures that all necessary information is captured. Users can confidently complete forms, knowing that 
they are guided through the process and not likely to overlook any fields, resulting in a more error-free experience.

Autofocus and the enter key functionality enhance the accessibility of the application for individuals who rely on 
keyboard navigation or assistive technologies. Users who prefer or require keyboard-based interactions can easily 
navigate through the application and complete forms without the need for mouse or touch input. This inclusivity ensures 
that the application is accessible to a wider range of users, including those with motor impairments or visual 
challenges.

By implementing these features, the application maintains consistency with user expectations, reducing the learning 
curve, ensuring a familiar and intuitive experience, and allure breathe life into the application's usability,
ushering users into a realm where efficiency reigns supreme.

\subsubsection{Relevant Changes}

\href{https://github.com/lyskouski/app-finance/issues/46}{https://github.com/lyskouski/app-finance/issues/46}

\subsection{Technical Details}

For a single field it's needed just `autofocus: true`-attribute while when we do want to control 
further steps to autofocus:

\begin{lstlisting}
class MyForm extends StatefulWidget {
  @override
  _MyFormState createState() => _MyFormState();
}

class _MyFormState extends State<MyForm> {
  late FocusNode _focusNode1;
  late FocusNode _focusNode2;
  late TextEditingController _controller1;
  late TextEditingController _controller2;

  @override
  void initState() {
    super.initState();
    _focusNode1 = FocusNode();
    _focusNode2 = FocusNode();
    _controller1 = TextEditingController();
    _controller2 = TextEditingController();
  }

  @override
  void dispose() {
    _focusNode1.dispose();
    _focusNode2.dispose();
    _controller1.dispose();
    _controller2.dispose();
    super.dispose();
  }

  @override
  Widget build(BuildContext context) {
    return Scaffold(
      appBar: AppBar(title: Text('Form Example')),
      body: Padding(
        padding: EdgeInsets.all(16.0),
        child: Column(
          children: [
            TextField(
              focusNode: _focusNode1,
              controller: _controller1,
              autofocus: true,
              decoration: InputDecoration(labelText: 'Field 1'),
              textInputAction: TextInputAction.next,
              onEditingComplete: () {
                FocusScope.of(context).requestFocus(_focusNode2);
              },
            ),
            TextField(
              focusNode: _focusNode2,
              controller: _controller2,
              decoration: InputDecoration(labelText: 'Field 2'),
              textInputAction: TextInputAction.done,
              onEditingComplete: () {
                // Perform any final actions here
              },
            ),
          ],
        ),
      ),
    );
  }
}
\end{lstlisting}

\subsection{Implementation Notes}

Proposal above might become overcomplicated for complex forms, so, we might take into account the usage of a state 
machine, also known as a finite-state machine (FSM). It's a conceptual model that is used to describe the behavior of a 
system in discrete states, transitions between those states, and actions associated with each state or transition.

For our case it should provide a scope of Focus Nodes and determined transition between them; to propagate it across
multiple widgets we'll use static class.

\begin{lstlisting}
class FocusController {
  // Define ignored case (when FocusNode is not defined)
  static const DEFAULT = -1;
  // List of generated nodes
  static List<FocusNode> nodes = [];
  // Index of field to focus on
  static int focus = DEFAULT;
  // Index to check previous focus
  static int _focus = DEFAULT;
  // Context is needed to request focus
  static late BuildContext _context;
  // BuildContext setter
  static void setContext(BuildContext context) {
    _context = context;
  }
  // Get focus node by index
  static FocusNode? getFocusNode(int idx) {
    // Generate nodes if missing
    while (idx >= nodes.length) {
      nodes.add(FocusNode());
    }
    return idx >= 0 ? nodes[idx] : null;
  }
  // For the last field set "done" state, all above - "next"
  static TextInputAction getAction(int idx) {
    return idx >= nodes.length ? TextInputAction.done : TextInputAction.next;
  }
  // Trigger re-focus on form
  static void requestFocus() {
    // Without delay focus event will concurrent with an update request
    Future.delayed(const Duration(milliseconds: 300), () {
      // To prevent multiple triggers for the same index
      if (focus >= 0 && _focus != focus) {
        _focus = focus;
        FocusScope.of(_context).requestFocus(nodes[focus]);
      }
    });
  }
  // Reset focus to search for a new one
  static void resetFocus() {
    focus = DEFAULT;
    _focus = DEFAULT;
  }
  // Used for `autofocus`-property on Widget
  static bool isFocused(int idx, dynamic value) {
    if ((value == null || value == '') &&
        idx != DEFAULT &&
        (focus == DEFAULT || focus == idx)) {
      focus = idx;
      requestFocus();
      return true;
    }
    return false;
  }
  // Cleanup
  static void dispose() {
    // Clone of `nodes` to avoid concurrent operations on list
    List<FocusNode> nodesCopy = List.of(nodes);
    for (FocusNode node in nodesCopy) {
      node.dispose(); // destroy Widget
      nodes.remove(node);
    }
    resetFocus();
  }
}
\end{lstlisting}

One of the problem that we might face with is an inability to open dropdowns anymore since 
`FocusScope.of(context).requestFocus` leads to re-build by any changed focus action. The problem is relevant to the
fact that we've been using `BuildContext` of our Form whereas it should be taken context from `DropdownButton`-element 
location. For the following example:

\begin{lstlisting}
Form(build `context1`) -> Widget(build `context2`) -> ... -> Widget(build `context3`) -> TextFormField.
\end{lstlisting}

\noindent We should trigger `FocusController.setContext(context);` directly for `context3`, otherwise re-render will 
block any additional interactions.

We're continuing bravely fix errors that producing by our own, and the next might be nor obvious, nor trivial till 
the fact how exactly `setState`-method should be used. So, we've been testing our application in build OC mode, but 
only once via Android emulator. The Flutter team tells us that everything will be working cross-operations but only 
in case we're using the functionality as it's been expected from their side. And the situation is that we're loosing 
the state of values that's been defined inside `StatefulWidget` class of our component that updated via 
`setState`-method inside `State`-extended class.

In our code, the issue with losing the form values state is related to the fact, that by assigning default values 
directly to the fields, those values will not be preserved when the widget is rebuilt. To maintain the form values 
state, we have to use the `initState`-method to initialize the fields with the provided values.

\begin{lstlisting}
class ExpensesTabState extends State<ExpensesTab> {
  String? account;
  String accountErrorMessage = '';
  String? budget;
  String budgetErrorMessage = '';
  Currency? currency;
  double? bill;
  String? description;
  DateTime? createdAt;

  @override
  void initState() {
    FocusController.resetFocus();
    account = widget.account;
    budget = widget.budget;
    currency = widget.currency;
    bill = widget.bill;
    description = widget.description;
    createdAt = widget.createdAt;
    super.initState();
  }
// ... other code
}
\end{lstlisting}


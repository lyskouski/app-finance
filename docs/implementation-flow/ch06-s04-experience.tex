% Copyright 2023 The terCAD team. All rights reserved.
% Use of this content is governed by a CC BY-NC-ND 4.0 license that can be found in the LICENSE file.

\subsection{Anticipating Experience}
\markboth{Optimizing}{Anticipating Experience}

Users have distinct expectations about how an application should generally behave ("know-how"), and addressing these 
expectations may be beneficial by demonstrating that the application aligns with user needs. This approach is known as 
user-centric design (UCD) \cite{Stil16}. UCD entails understanding the needs, goals, and pain points of the target 
audience through user research, creating user personas, and gathering feedback.

We should design interfaces that resonate with users and meet their expectations. Consistent design elements, such as 
button placement, color schemes, and typography, help users feel at ease and reduce their cognitive load.


\subsubsection{Adding Localizations} \label{locale}

Adapting to different languages and regions makes the application more accessible and inclusive to a global audience. 
The quality of the translation significantly impacts this accessibility \cite{Lomm07}. This involves translating text, 
adjusting layouts for different languages, and incorporating region-specific content and functionalities.

\noindent Localization can be enabled from a configuration file, \q{pubspec.yaml}, with an additional declaration in 
\q{l10n.yaml}-file: 

\begin{lstlisting}[language=yaml]
# ./pubspec.yaml
dependencies:
  flutter_localizations:
    sdk: flutter

# ./l10n.yaml
arb-dir: lib/l10n
template-arb-file: app_en.arb
output-localization-file: app_localization.dart
preferred-supported-locales:
  - en
  - be
\end{lstlisting}

\noindent The \q{MaterialApp}-widget should be extended by \q{locale}-properties \issue{7}{}:

\begin{lstlisting}
// Autogenerated package from '.arb'-files
import 'package:app_finance/l10n/app_localization.dart';
// Main application builder
Widget build(BuildContext context) => MaterialApp(
  localizationsDelegates: AppLocalizations.localizationsDelegates,
  // List is generated based on the '.arb'-files availability
  supportedLocales: AppLocalizations.supportedLocales,
  // 'AppLocale' extends from 'ValueNotifier', returns 'Locale'
  locale: context.watch<AppLocale>().value,
  // ... other options
);
\end{lstlisting}

\noindent In the context of \q{Apple}-based applications, it's necessary to include a localization section in 
\q{Runner/Info.plist}-files:

\begin{lstlisting}[language=xml]
<dict>
  <key>CFBundleLocalizations</key>
  <array>
    <string>en</string>
    <!-- other languages -->
  </array>
\end{lstlisting}

\noindent Rich text can be stored in the \q{assets}-folder as \q{.md}-files and integrated back into an application 
using the \q{flutter\_markdown}-package:

\begin{lstlisting}
FutureBuilder(
  future: DefaultAssetBundle.of(context).loadString(
    './assets/l10n/$fileName.md'
  ),
  builder: (context, AsyncSnapshot<String> snapshot) =>
    snapshot.hasData ? Markdown(data: snapshot.data!) : Container(),
);
\end{lstlisting}

\noindent Localization encompasses more than just translating text \cite{Hofs03}; it involves adapting content and 
design for different regions and cultures. For instance, in Japan, people usually get information through stories, while 
in Germany, information should be complete, clear, and precise. For example, consider adjusting the layout by 
rearranging elements from left-to-right to right-to-left, such as repositioning a left-hand navigation bar to the right, 
for languages like Arabic, Hebrew, and Japanese. In Germany, a navigation bar is usually arranged alphabetically, 
whereas in most other regions, it is organized by usage intensity priority. In France, the typical orientation is 
centered, meaning that buttons are usually located in the center of the menu bar rather than in a corner.

Asian cultures often prefer vibrant layouts that emphasize the importance of contextual relationships between elements. 
In contrast, Western cultures tend to prefer minimalist designs with a limited color palette that place primary emphasis 
on key objects or elements.

This contrast stems from the fundamental difference in cultural norms between collectivism and individualism 
\cite{Wall23}. Individualist cultures prioritize autonomy and personal goals, while collectivist cultures emphasize 
interdependence and community goals. These cultural distinctions are manifested in the design of digital products. For 
example, Canadian apps tend to be minimalist and lack social features, whereas Indian apps often incorporate social 
support functions.

Icons are visual metaphors that represent user actions, and their interpretation can vary significantly across cultures. 
For example, when using icons to indicate currencies, locations, or navigation, it's important to be mindful that 
symbols may not have universal meanings and can be understood differently. For instance, Chinese apps don't use 
hamburger or kebab symbols for the "menu" button. Western user interfaces typically emphasize a single call to action, 
while Asian interfaces often feature multiple calls to action. This indicates that collective cultures incorporate a 
greater number of visual and interactive elements than individualist cultures do.

Differences in visual identity may stem from contrasts between masculine cultures, like the United States and Italy, and 
feminine cultures, such as Thailand and the Netherlands. The former group is characterized by the accelerated 
presentation of information, which facilitates effective decision-making. The latter group emphasizes cultivating an 
emotional connection to the subject matter, primarily through the use of infographics \cite{Wal923}.

Color preferences can vary widely. Europeans tend to favor cooler colors, while Latin Americans prefer warmer hues. 
Asian cultures embrace vibrant and diverse color palettes.

For example, the Facebook registration page exhibits variations in layout and color schemes based on the
target audience (\cref{img:ui-facebook}). In collectivist regions, the layout is more vibrant, featuring a blue 
background and a more prominent call-to-action button. In contrast, individualist regions display a minimalist design 
with a white background and a less conspicuous call-to-action button.

\img{uiux/fb-localized}{Facebook Registation Page Deviations}{img:ui-facebook}

\noindent These details \cite{Rein14} contribute to an improved user experience for diverse audiences, underscoring the 
importance of localization in collectivist regions. Neglecting localization in such areas could result in a higher 
churn rate.


\subsubsection{Creating Navigational Clarity} 

Designing clear and logical navigation paths, minimizing complexity, and ensuring that users can seamlessly move through 
the application's features and content are all part of asserting user experience.

There should be a minimal set of actions to achieve the goal. As an example, by showing additional actions by swiping 
on element (\cref{img:u-swipe}). In our case, that helps to access \q{Edit}-form or \q{Delete} item without going 
through the multiple navigation steps. Such behavior can be achieved by the usage of 
\q{flutter\_swipe\_action\_cell}-component as a wrapper of our Widget \issue{206}{}.

To accommodate different screen resolutions, the presentation of navigation layers may vary. This adaptation ensures a 
consistent and user-friendly experience across various devices and screen sizes (\cref{img:u-nav}). For instance, on 
larger screens, a persistent navigation bar may be displayed, while on smaller screens, a collapsible menu or bottom 
navigation bar might be more appropriate. This flexibility allows users to easily access navigation options regardless 
of the device they are using.

\img{uiux/swipe-actions}{Swipe Actions on Cell}{img:u-swipe}

\img{uiux/navigation-bar}{Navigation Bar deviations across multiple screens}{img:u-nav}


\subsubsection{Personalizing Layout}

Tailored content recommendations and customizable settings enable applications to improve the user experience by 
providing more personalized and engaging interactions.

Users expect not only functionality, but also the ability to customize their experience. One aspect of customization is 
the ability to configure layout settings, such as color schemes \issue{231}{}, font sizes \issue{237}{}, and content 
arrangement \issue{258}{}. Allowing users to configure these settings empowers them to create an interface that suits 
their needs and aesthetic preferences (see \cref{img:u-custom}). This reduces cognitive load and makes navigation more 
intuitive and personalized.

\img{uiux/layout-editor}{Layout Configuration}{img:u-custom}

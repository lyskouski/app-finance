% Copyright 2023 The terCAD team. All rights reserved.
% Use of this content is governed by a CC BY-NC-ND 4.0 license that can be found in the LICENSE file.

\subsection{[TBD] Dart}
\markboth{Bootcamping}{Dart}

\textbf{Dart}-language, by following Object-Oriented Programming (OOP) paradigm, enables extending even built-in data 
types (from \q{String}, \q{double}, ... to \q{null}; \q{dynamic} - calculated / changeable type at run-time):

\begin{lstlisting}
int counter = 123; // counter.toDouble() => 123.0
double pi = 3.141592; // pi.clamp(0, 3) => 3.0
bool isVisible = true; // true.toString() => 'true'
String text = 'Some content'; // text.codeUnits // List<int>
// 'StringBuffer' as a way to construct strings sequentially
final spread = StringBuffer();
spread.write('some text...'); // concatenate
spread.writeln('... another content'); // add indentation at the end
// But long text can be also broken and implicitly spliced back
String test = 'some long text...'
  '...continuation';
// Convert other types to String by interpolation (via '$' sign)
String sample = '$counter $isVisible';
// Extend functionality of 'double'-type
extension DoubleExt on double {
  // 0.12.pcnt => '12.0%'
  String get pcnt => '${this * 100}%';
}
\end{lstlisting}

\noindent Collections are represented by \q{List}, \q{Map}, and \q{Set}. Where \q{List} is an ordered collection of 
objects; \q{Map}, as a collection of key/value pairs, is used to retrieve a value by its associated key with a 
maintained key uniqueness; as well as \q{Set} -- to control uniqueness of variables. As a distinctive feature, 
iterativeness for \q{Map} deviates by its type:

\begin{itemize}
  \item \q{HashMap}, unordered, provides access to items (by key) in (potentially) amortized constant time;
  \item \q{LinkedHashMap} iterates in key insertion order;
  \item \q{SplayTreeMap} is based on a self-balancing binary tree that allows most single-entry operations in amortized 
  logarithmic time;
  \item Immutable Map (in round brackets) is not iterable but simplify the access to its values and control their
  persistency in \q{List}.
\end{itemize}

\begin{lstlisting}
List<int> data = [1, 2, 3]; // data.last; // 3
Set<String> scope = {'a', 'b', 'a'}; // scope.length; // 2
var map = (text: 'sample', at: 123); // map.text; // 'sample'
HashMap<String, int> hash = {'test': 123}; // hash.values; // [123]
\end{lstlisting}


\subsubsection{Overloading operators}

"Magic methods" are often referred to as "operator overloading" or "special methods." They allow to declare a custom 
behavior for built-in operations:

\begin{itemize}
  \item \q{toString} returns a string representation of an object, can be used for a serialization and deserialization 
  process of an object;
  \item \q{call} allows an object to be treated as a function;
  \item \q{hashCode} returns a hash code for an object (to use it as a key for \q{Map} and \q{Set}, and to override \q{==});
  \item \q{operator==} compares two objects for equality;
  \item \q{get} and \q{set} -- to override the behavior of getting and setting properties.
\end{itemize}

\begin{lstlisting}
class Person {
  // Required from constructor
  String name;
  // Post-initialization
  late DateTime _createdAt = DateTime.now();
  // var person = Person('Tom');
  Person(this.name);
  // person() // 'Hello from Tom!'
  String call() => 'Hello from $name!';
  // person.createdAt = DateTime(2023, 01, 01);
  set createdAt(DateTime date) => _createdAt = date;
  // print(person.createdAt); // 2023-01-01 00:00:00
  DateTime get createdAt => _createdAt;
  // print(Person('Tom') == Person('Terry')); // false
  @override
  int get hashCode => int.parse(name.codeUnits.join(''));
  @override
  bool operator ==(Object other) => other is Person && other.name == name;
  // person = Person.fromString('Tom');
  factory Person.fromString(String name) {
    return Person(name);
  }
  // print(person); // 'Tom'
  @override
  String toString() => name;
}
\end{lstlisting}


\subsubsection{Declaring Input Arguments}

A few options are available for the arguments declaration:

\begin{lstlisting}
// Ordered scope
/// Sample: add('test', null) 
/// Sample: add('test', 123) 
void add(String value, int? id);

// With optional arguments
/// Sample: add('test')
void add(String value, [int? id]);
void add(String value, [int id = 123]); // preset for 'id'

// Named attributes
/// Sample: add(value: 'test')
/// Sample: add(id: 1, value: 'test')
void add({String value, int? id});
void add({String value, int id = 123});

// Mix
/// Sample: add('test', id: 123)
void add(String value, {int? id});
\end{lstlisting}

\noindent In the context of classes, their properties can be protected (not declared in constructor, accessible for 
extends), private (with underscore; neither visible, nor accessible from extends), and public (mentioned in 
constructor); with extra options as \q{final} (immutable), \q{const} (not changeable for run-time), \q{static} (to 
grant access without object initialization), and \q{late} (postponed initialization):

\begin{lstlisting}
class Person {
  // Only static fields can be declared as const
  static const gender = 'unknown';
  // Private property
  String _priv;
  // Immutable after an object creation
  final String name;
  // Delayed assignment
  late final int age;
  // Post-initialization for a dynamic content
  late cast = PersonCast(age);
  // constructor
  Person(
    this.name, {
    int? age
    }
  // Post-processing (random age if not set)
  ) : this.age = age ?? Random().nextInt(120);
}
\end{lstlisting}


\subsubsection{Casting Dynamic Structures}

It might be useful to not declare the proper type:

\begin{lstlisting}
extension StringExt on String {
  // Any letter in uppercase highlights that
  // type would be set during the call
  /// Sample: '[1,2,3]'.toList<int>(); // [1, 2, 3] 
  List<T> toList<T>() {
    final data = length > 0 ? json.decode(this) : [];
    // Convert to declared Type
    return data.cast<T>();
  }
}
\end{lstlisting}

\noindent Or, use \q{dynamic} instead of \q{T} to handle different types simultaneously:

\begin{lstlisting}
extension StringExt on String {
  /// Sample: '[1,"2",3.0]'.toList(); // [1, '2', 3.0] 
  List<dynamic> toList() =>
      length > 0 ? json.decode(this) : [];
}
\end{lstlisting}


\subsubsection{Asserting Functions Definition}

By using a type definition (\q{typedef}) we can simplify representation of a complex structure as arguments or 
declare the expected structure of a function (propagated as an argument):

\begin{lstlisting}
typedef SetViewFunction = String Function(Currency input);

typedef Nested = Map<String, List<Person>>;

class CurrencySelector {
  Nested key;
  SetViewFunction? setView;

  CurrencySelector({
    this.key,
    this.setView,
  }) {
    // Set if 'null'
    setView ??= setDefaultView;
  }

  String setDefaultView(Currency input) {
    // Default implementation
  }
}
\end{lstlisting}

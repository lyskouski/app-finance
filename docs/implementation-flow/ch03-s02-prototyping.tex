% Copyright 2023 The terCAD team. All rights reserved.
% Use of this content is governed by a CC BY-NC-ND 4.0 license that can be found in the LICENSE file.

\subsection{[TBD] Prototyping Application Skeleton}
\markboth{Implementing Core Functionality}{Prototyping Application Skeleton}

[TBD]

\subsubsection{Creating Unit Tests}

By now we've created some business logic functionality with not so trivial flow. Not to waste the time on checking it
all the time manually after significant changes or refactoring, let's create some tests to check it for us automatically.

Furthermore we've already broken our first test (that's been provided by a Flutter template generation). At the 
beginning it was covering the `Add`-button as an increment trigger for the shown counter.

\begin{lstlisting}
void main() {
  testWidgets('Counter increments smoke test', (WidgetTester tester) async {
    // Build our app and trigger a frame.
    await tester.pumpWidget(const MyApp());

    // Verify that our counter starts at 0.
    expect(find.text('0'), findsOneWidget);
    expect(find.text('1'), findsNothing);

    // Tap the '+' icon and trigger a frame.
    await tester.tap(find.byIcon(Icons.add));
    await tester.pump();

    // Verify that our counter has incremented.
    expect(find.text('0'), findsNothing);
    expect(find.text('1'), findsOneWidget);
  });
}
\end{lstlisting}

\noindent Let's check that `Add`-button is opening a Widget with a three named tabs.

\begin{lstlisting}
void main() {
  testWidgets('Given Main page When tap on Create Then opened BillAddPage', (WidgetTester tester) async {
    // App initialization
    await tester.pumpWidget(
        // Taken as is from `void main` of `lib/main.dart`-file
        ChangeNotifierProvider(
            create: (_) => AppData(),
            child: const MyApp(),
        )
    );
    // Checking one of the text widgets
    expect(find.text('Goals'), findsOneWidget);
    // Clicking on `Add`-button
    await tester.tap(find.byIcon(Icons.add));
    await tester.pumpAndSettle(); // Waiting till finished render state
    // Asserting icons on our tabs
    expect(find.byIcon(Icons.insert_invitation), findsOneWidget);
    expect(find.byIcon(Icons.money_off), findsOneWidget);
    expect(find.byIcon(Icons.transform), findsOneWidget);
  });
}
\end{lstlisting}

Here we are, by `tab` and `expect` the flow of the whole application can be checked. Such type of tests are named as 
Widget tests that emphasized by specially named `testWidgets`-function.

Initially, it would be better to concentrate mostly on Business Logic than on the draft state of our pages. We'll 
return later to UI/UX, redesign if not dramatically but vastly, and right after that write a reliable 
Test Suite for the Widgets. To check business based calculations it's used `test`-function, and, by example, we would
take our implemented Account Recalculation flow.

\begin{lstlisting}
void main() {
  group('Account Recalculation:getDelta', () { // Add prefix to all the tests inside
    late AccountRecalculation mock; // Deferred initialization 
    // `setUp`-is going to be executed before each(!) test
    setUp(() {
      final accountMock = AccountAppData(
        uuid: '1',
        title: 'test',
        type: AppAccountType.account.toString(),
      );
      mock = AccountRecalculation(
        initial: accountMock.clone(),
        change: accountMock.clone(),
      );
    });
    // Define "table" of an input arguments and output assertions
    final testCases = [
      [false, 5.0, false, 1.0, -4.0],
      [false, 1.0, false, 5.0, 4.0],
      [true, 2.0, false, 3.0, 3.0],
      [false, 2.0, true, 3.0, -2.0],
    ];
    // Run all defined Test Cases
    for (var v in testCases) {
      test('initial(hidden[${v[0]}], details[${v[1]}]), change(hidden[${v[2]}], details[${v[3]}]) => ${v[4]}', () {
        // Update inputs
        mock.initial!.hidden = v[0] as bool;
        mock.initial!.details = v[1] as double;
        mock.change.hidden = v[2] as bool;
        mock.change.details = v[3] as double;
        // Assert `AccountRecalculation.getDelta`-method results
        expect(mock.getDelta(), v[4]); 
      });
    }
  });
}
\end{lstlisting}

By typing `flutter test` in the command line we would see something like that:

\begin{lstlisting}
[+] Given Main page When tap on Create Then opened BillAddPage
[+] Account Recalculation:getDelta initial(hidden[false], details[5.0]), change(hidden[false], details[1.0]) => -4.0
[+] Account Recalculation:getDelta initial(hidden[false], details[1.0]), change(hidden[false], details[5.0]) => 4.0
Expected: <3.0>
  Actual: <1.0>
[X] Account Recalculation:getDelta initial(hidden[true], details[2.0]), change(hidden[false], details[3.0]) => 3.0
[+] Account Recalculation:getDelta initial(hidden[false], details[2.0]), change(hidden[true], details[3.0]) => -2.0
Null check operator used on a null value
[X] Account Recalculation:getDelta initial(null), change(hidden[false], details[2.0]) => 2.0
Exited (1)
\end{lstlisting}

That test has highlighted a dozen of troubles and failures in our code, so, it definitely has not been written in vain.

\begin{lstlisting}
double getDelta() {
  return change.hidden
-  ? -initial!.details
+  ? -(initial?.details ?? 0.0)
-  : change.details - initial!.details;
+  : (initial?.hidden ?? true
+      ? change.details
+      : change.details - initial?.details);
}
\end{lstlisting}
% Copyright 2023 The terCAD team. All rights reserved.
% Use of this content is governed by a CC BY-NC-ND 4.0 license that can be found in the LICENSE file.

\subsection{Visualizing Data}
\markboth{Unleashing}{Visualizing Data}

With countless transactions, revenue streams, expenditures, and budgetary considerations, it's essential to have a 
system in place that simplifies the intricacies. The traditional reliance on spreadsheets and numerical reports to 
handle this complexity often results in a daunting and overwhelming experience, especially as data volumes continue 
to grow.  This challenge necessitates solutions for visualizing and interpreting data effectively. Our financial 
accounting application may employ the data visualization in various use cases:

\begin{itemize}
  \item Cash Flow Analysis -- predicts financial trends, manage liquidity, and ensure operational continuity.

  \item Budget Monitoring -- helps to stay on track and avoid overspending.

  \item Financial Reporting -- provides summary reports, including annual reports, income statements, and balance 
  sheets, offering a snapshot of transactions within a defined period.

  \item Risk Management -- visualizes risk factors and their impact to make proactive decisions in mitigating financial 
  threats.

  \item Portfolio Management -- assess the performance of assets for informed investment decisions.
\end{itemize}

\noindent While these categories can encompass a wide range of tools and techniques, our initial focus is on delivering 
high-impact features shortly (since the full scope might take months). As we move forward, we'll progressively enhance 
the application by introducing a broader scope of options.

Let's kick things off by introducing a first chart for managing budget categories: Forecast Chart. 
This chart provides a visual representation of historical data while also forecasting the future based on mathematical 
models. These models can vary depending on the specific field and the type of data being analyzed. We'll take the 
simplest one -- Monte-Carlo Simulation, a computational technique used to approximate the behavior of complex systems 
or processes through random sampling. This simulation method is particularly valuable when dealing with systems that 
involve many variables and intricate relationships. By simulating numerous scenarios through random sampling, Monte 
Carlo simulations provide a way to explore a wide range of possibilities and assess the likelihood of different 
financial scenarios:

\begin{lstlisting}
// ./lib/_classes/math/monte_carlo_simulation.dart
class MonteCarloSimulation {
  final Random rnd = Random();
  final int cycles;
  final double coercion;
  MonteCarloSimulation({this.cycles = 30, this.coercion = 1});
  List<Offset> generate(List<Offset> data, num step, num max) {
    final List<List<double>> distribution = [];
    // Loop through the scope (provided data points)
    for (int i = 0; i < data.length; i++) {
      final state = mcNormal(data[i].dy, coercion, cycles);
      // Loop through the states generated for each data point
      for (int j = 0; j < state.length; j++) {
        if (j >= distribution.length)
          distribution.add([]);
        distribution[j].add(state[j]);
      }
    }
    double posX = data.last.dx + step;
    List<Offset> result = [];
    int idx = 0;
    // Generate simulated data points for the forecast
    while (posX <= max) {
      result.add(Offset(
        posX, 
        distribution[idx][
          distribution[idx].length * rnd.nextDouble() ~/ 1],
          // where '~/ 1' is equal to '.toInt()'
      ));
      posX += step;
      idx++;
    }
    return result;
  }
  // To generate a list of random values using the simulation
  List<double> mcNormal(double mean, double stdDev, int samples) {
    List<double> results = [];
    // Perform the Monte Carlo simulation 
    // for the specified number of samples
    for (int i = 0; i < samples; i++) {
      results.add(_normalRandom(mean, stdDev));
    }
    return results;
  }
  // A random value based on a normal distribution
  double _normalRandom(double mean, double stdDev) {
    double u1 = rnd.nextDouble();
    double u2 = rnd.nextDouble();
    double z0 = sqrt(-2.0 * log(u1)) * cos(2 * pi * u2);
    return mean + stdDev * z0;
  }
}
\end{lstlisting}

\noindent It's left to generate the chart itself \issue{4}{}, which can be achieved using the \q{CustomPaint}-widget. 
The advantage of using this widget is that it allows us to separate the chart's line (\q{painter}-property) from a
secondary data like axes and background colors (\q{foregroundPainter}-property):

\begin{lstlisting}
// ./lib/charts/forecast_chart.dart
final xMin = DateTime(now.year, now.month);
final xMax = DateTime(now.year, now.month + 1);
final bg = ForegroundChartPainter(
  yMin: 0.0,
  yMax: 140,
  xMin: xMin.millisecondsSinceEpoch.toDouble(),
  xMax: xMax.millisecondsSinceEpoch.toDouble(),
  yArea: [80, 120], // green area as a threshold
);
return SizedBox(
  height: size.height,
  width: size.width,
  child: CustomPaint(
    size: size,
    painter: ForecastChartPainter(
      data: data,
      yMax: yMax * bg.yMax / 100,
      xMin: xMin.millisecondsSinceEpoch.toDouble(),
      xMax: xMax.millisecondsSinceEpoch.toDouble(),
    ),
    foregroundPainter: bg,
    willChange: false, // to avoid re-build
    child: Padding(
      padding: EdgeInsets.only(top: indent / 4),
      child: Text(tooltip),
    ),
  ),
);
\end{lstlisting}

\noindent We have created our own painters, named \q{ForegroundChartPainter} (gonna be used for all other charts) and 
\q{ForecastChartPainter}. These classes are responsible for plotting a graphical information on the \q{Canvas}:

\begin{lstlisting}
// ./lib/charts/painter/forecast_chart_painter.dart
class ForecastChartPainter extends CustomPainter {
  // ... properties declaration
  ForecastChartPainter({ /* ... */ });

  @override
  void paint(Canvas canvas, Size size) {
    for (final scope in data) {
      // Plot historical data
      _paint(canvas, scope.data, size, scope.color);
      final dx = scope.data.last.dx;
      final total = _sumY(scope.data);
      if (scope.data.length > 2 && dx < xMax && total < yMax) {
        final cycles = (xMax - dx) ~/ msDay;
        final forecast = [Offset(scope.data.last.dx, total)];
        forecast.addAll(MonteCarloSimulation(cycles: cycles).generate(scope.data, msDay, xMax - 2 * msDay));
        // Draw forecast line
        _paint(canvas, forecast, size, scope.color.withBlue(200).withOpacity(0.4));
      }
    }
  }

// ... other stuff

  // Draw point to reflect data
  void _paintDot(Canvas canvas, Offset point, Color color) {
    final dot = Paint()..color = color;
    canvas.drawCircle(point, 2.2, dot);
  }
  // Draw a curve using four points
  _paintCurve(Canvas canvas, Offset p0, Offset k1, Offset k2, Offset p1, Color color) {
    final line = Paint()
      ..color = color
      ..style = PaintingStyle.stroke
      ..strokeWidth = 2;
    final path = Path()..moveTo(p0.dx, p0.dy);
    path.cubicTo(k1.dx, k1.dy, k2.dx, k2.dy, p1.dx, p1.dy);
    canvas.drawPath(path, line);
  }
}
\end{lstlisting}

\noindent That's a wrap! Following the implementation of various charts, including radial bar \issue{128}{}, column 
and bar race charts \issue{147}{}, OHLC (open, high, low, close) chart \issue{148}{}, gauge chart \issue{160}{} 
\issue{180}{}, pie chart \issue{179}{}, currency trades \issue{182}{};, we've curated a set of valuable metrics. 
These metrics (\cref{img:f-charts}) empower users to gain deeper insights into their financial situations.

\img{features/charts}{Visualization of Charts}{img:f-charts}
